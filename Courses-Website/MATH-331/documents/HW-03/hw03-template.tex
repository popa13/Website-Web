\documentclass[12pt]{article}
\usepackage[utf8]{inputenc}

\usepackage{enumitem}
\usepackage[margin=2cm]{geometry}

\usepackage{amsmath, amsfonts, amssymb}
\usepackage{graphicx}
\usepackage{tikz}
\usepackage{pgfplots}
\usepackage{multicol}

\usepackage{comment}
\usepackage{url}

\usepackage{titlesec}
\titleformat{\section}[frame]
{\normalfont\scshape}
{\thesection}{8pt}{\centering}

\usepackage{array}

\pgfplotsset{compat=1.16}

%\usepackage[margin=1cm]{cloze}

\usepackage[thmmarks]{ntheorem}

% MATH commands
\newcommand{\bC}{\mathbb{C}}
\newcommand{\bR}{\mathbb{R}}
\newcommand{\bN}{\mathbb{N}}
\newcommand{\bZ}{\mathbb{Z}}
\newcommand{\bT}{\mathbb{T}}
\newcommand{\bD}{\mathbb{D}}

\newcommand{\cL}{\mathcal{L}}
\newcommand{\cM}{\mathcal{M}}
\newcommand{\cP}{\mathcal{P}}
\newcommand{\cH}{\mathcal{H}}
\newcommand{\cB}{\mathcal{B}}
\newcommand{\cK}{\mathcal{K}}
\newcommand{\cJ}{\mathcal{J}}
\newcommand{\cU}{\mathcal{U}}
\newcommand{\cO}{\mathcal{O}}
\newcommand{\cA}{\mathcal{A}}
\newcommand{\cC}{\mathcal{C}}

\newcommand{\fK}{\mathfrak{K}}
\newcommand{\fM}{\mathfrak{M}}

\newcommand{\ga}{\left\langle}
\newcommand{\da}{\right\rangle}
\newcommand{\oa}{\left\lbrace}
\newcommand{\fa}{\right\rbrace}
\newcommand{\oc}{\left[}
\newcommand{\fc}{\right]}
\newcommand{\op}{\left(}
\newcommand{\fp}{\right)}

\newcommand{\ra}{\rightarrow}
\newcommand{\Ra}{\Rightarrow}

\renewcommand{\Re}{\mathrm{Re}\,}
\renewcommand{\Im}{\mathrm{Im}\,}
\newcommand{\Arg}{\mathrm{Arg}\,}
\newcommand{\Arctan}{\mathrm{Arctan}\,}
\newcommand{\sech}{\mathrm{sech}\,}
\newcommand{\csch}{\mathrm{csch}\,}
\newcommand{\Log}{\mathrm{Log}\,}
\newcommand{\cis}{\mathrm{cis}\,}

\newcommand{\ran}{\mathrm{ran}\,}
\newcommand{\bi}{\mathbf{i}}
\newcommand{\Sp}{\mathrm{span}\,}
\newcommand{\Inv}{\mathrm{Inv}\,}
\newcommand\smallO{
  \mathchoice
    {{\scriptstyle\mathcal{O}}}% \displaystyle
    {{\scriptstyle\mathcal{O}}}% \textstyle
    {{\scriptscriptstyle\mathcal{O}}}% \scriptstyle
    {\scalebox{.7}{$\scriptscriptstyle\mathcal{O}$}}%\scriptscriptstyle
  }
\newcommand{\HOL}{\mathrm{Hol}}
\newcommand{\cl}{\mathrm{clos}}
\newcommand{\ve}{\varepsilon}

\tikzstyle{myboxT} = [draw=black, fill=black!0,line width = 1pt,
    rectangle, rounded corners = 0pt, inner sep=8pt, inner ysep=8pt]
    
\newcommand{\MyC}[1]{\begin{tikzpicture}
\node (boxIntro) at (0,0) {};
\node [myboxT](Intro) at (boxIntro){%
	\begin{minipage}{0.9\textwidth}
	#1
	\end{minipage}};
\end{tikzpicture}}

%%%%  Environnement exer et solutionnaire
{\theorembodyfont{}
\theoremstyle{plain}
\theoremseparator{\textbf{.}}
\theoremsymbol{}
\newtheorem{exer}{\textbf{Exercise}}}

{\theorembodyfont{\color{blue}}
\theoremstyle{plain}
\theoremseparator{\textbf{:}}
\theoremsymbol{$\square$}
\newtheorem*{sol}{\textbf{Solution}}}


\renewcommand*{\theexer}{\arabic{exer}}
\renewcommand*{\thesol}{\arabic{sol}}


\newcommand{\headHW}[4]{%
	\noindent \hrulefill \\
	MATH-#1 #2 \\
	#3 #4
	
}

\begin{document}
	\noindent \hrulefill \\
	MATH-331 Introduction to Real Analysis \hfill YOUR FULL NAME\\
	Homework 03 \hfill Fall 2021\\\vspace*{-0.7cm}
	
	\noindent\hrulefill
	
	\noindent Due date: October 11${}^{\text{th}}$ 1:20pm \hfill Total: \hspace{0.3cm}/70.
	
\vspace*{0.5cm}

	\bgroup \renewcommand{\arraystretch}{1.5}
\begin{table}[h]
\centering
\begin{tabular}{|m{1.5cm}|>{\centering\arraybackslash}p{0.75cm}|>{\centering\arraybackslash}p{0.75cm}|>{\centering\arraybackslash}p{0.75cm}|>{\centering\arraybackslash}p{0.75cm}|>{\centering\arraybackslash}p{0.75cm}|>{\centering\arraybackslash}p{0.75cm}|>{\centering\arraybackslash}p{0.75cm}|>{\centering\arraybackslash}p{0.75cm}|>{\centering\arraybackslash}p{0.75cm}|>{\centering\arraybackslash}p{0.75cm}|}
\hline
Exercise & 1 (5) & 2 (5) & 3 (5) & 4 (5) & 5 (10) & 6 (10) & 7 (5) & 8 (5) & 9 (5) & 10 (10) \\
\hline
Score & & & & & & & & & &  \\\hline
\end{tabular}
\caption{Scores for each exercises}
\end{table}
\egroup
	
\vspace*{0.5cm}

{\bf Instructions:} You must answer all the questions below and send your solution by email (to \url{parisepo@hawaii.edu}). If you decide to not use {\LaTeX} to hand out your solutions, please be sure that after you scan your copy, it is clear and readable. Make sure that you attached a copy of the homework assignment to your homework. 

\noindent If you choose to use {\LaTeX}, you can use the template available on the course website.

\noindent No late homework will be accepted. No format other than PDF will be accepted. Name your file as indicated in the syllabus.

\section{Writing problems}
For each of the following problems, you will be asked to write a clear and detailed proof. You will have the chance to rewrite your solution in your semester project after receiving feedback from me.

%% ---------------------------------------------
\begin{exer}
(5 pts)
Let $(a_n)_{n = 1}^\infty$ be an increasing sequence and $(b_n)_{n = 1}^\infty$ be a decreasing sequence. Let $(c_n)_{n = 1}^\infty$ be the sequence defined by $c_n = b_n - a_n$. Show that if $\lim_{n \ra \infty} c_n = 0$, then the sequences $(a_n)_{n = 1}^\infty$ and $(b_n)_{n = 1}^\infty$ converges and $\lim_{n \ra \infty} a_n = \lim_{n \ra \infty} b_n$.
\end{exer}
\begin{sol}
Your solution here.
\end{sol}

%-----------------------------------------------
\begin{exer}
(5 pts)
Let $f: D \subseteq \bR \ra \bR$, and suppose that $x_0$ is an accumulation point of $D$. Suppose that for each sequence $(x_n)_{n = 1}^\infty$ converging to $x_0$ with $x_n \in D\backslash \{ x_0 \}$ for each $n \geq 1$, then the sequence $(f(x_n))_{n = 1}^\infty$ is Cauchy. Show that $f$ has a limit at $x_0$.

[Hint: For two sequences $(x_n)$ and $(y_n)$ that satisfy the assumption, define the sequence $(z_n)$ to be $z_{2n} = x_n$ and $z_{2n - 1} = y_n$. Show that $(f(z_n))$ converges and the sequence $(f(x_n))$ and $(f(y_n))$ converges to the same limit as $(f(z_n))$. Conclude by a theorem in the lecture notes.]
\end{exer}

%-----------------------------------------------
\begin{exer}
(5 pts)
Prove that if $f : D \subseteq \bR \ra \bR$ has a limit at $x_0 \in \mathrm{acc}\, D$, then the limit is unique.
\end{exer}

%-----------------------------------------------
\begin{exer}
(5 pts)
Suppose $f: D \subseteq \bR \ra \bR$, $g: D \subseteq \bR \ra \bR$ and $h : D \subseteq \bR \ra \bR$ are three functions such that
	\begin{align*}
	f(x) \leq h(x) \leq g(x) \quad (\forall x \in D ) .
	\end{align*}
Suppose that $f$ and $g$ have limits at $x_0$ with $\lim_{x \ra x_0} f(x) = \lim_{x \ra x_0} g(x)$. Prove that $h$ has a limit at $x_0$ and
	$$
	\lim_{x \ra x_0} f(x) = \lim_{x \ra x_0} h(x) = \lim_{x \ra x_0} g(x) .
	$$
\end{exer}

%-----------------------------------------------
\begin{exer}
(10 pts)
Let $f : (0, \infty ) \ra \bR$ be a function. We say that $f$ has a limit at $\infty$ if there exists a $L \in \bR$ such that for any $\varepsilon > 0$, there is a real number $M > 0$ such that if $x > M$, then $|f(x) - L| < \varepsilon$. 
	\begin{enumerate}[label=\textbf{\alph*)}]
	\item Show that if $g: (0, \infty ) \ra \bR$ is bounded and $\lim_{x \ra \infty} f(x ) = 0$, then $\lim_{x \ra \infty} f(x) g(x) = 0$.
	\item Let $a > 0$ and suppose that $f: (a, \infty ) \ra \bR$ and define $g : (0, 1/a ) \ra \bR$ by $g(x) = f(1/x)$. Show that $f$ has a limit at $\infty$ if and only if $g$ has a limit at $0$.
	\end{enumerate}
\end{exer}

\section{Homework problems}
Answer all the questions below. Make sure to show your work.

\begin{exer}
(10pts)
For each of the sequences below, determine its nature (converges or diverges)\footnote{You don't need to compute the limit.}:
	\begin{enumerate}[label=\textbf{\alph*)}]
	\item $(a_n)$ where $a_n = \frac{1}{n} + \frac{1}{n + 1} + \cdots + \frac{1}{2n}$.
	\item $(a_n)$ where $a_n = \frac{1 + 2 + \cdots + n}{n^2}$.
	\end{enumerate}
\end{exer}

%------------------------------------------------
%\begin{exer}
%Define $f : (-2, 0) \ra \bR$ by $f(x) = \frac{2x^2 + 3x - 2}{x + 2}$. Prove that $f$ has a limit at $x_0 = -2$, and find it.
%\end{exer}

%------------------------------------------------
\begin{exer}
(5 pts)
Define $g: (0, 1) \ra \bR$ by $f(x) = \frac{\sqrt{1 + x} - 1}{x}$. Prove that $g$ has a limit at $0$ and find it.
\end{exer}

%------------------------------------------------
\begin{exer}
(5 pts)
Suppose that $f: (0, 1) \ra \bR$ has a limit at $x_0 = 1$ and $\lim_{x \ra 1} f(x) = 1$. Compute the value of the limit
	\begin{align*}
	\lim_{x \ra 1} \frac{f(x) (1 - f(x)^2)}{1 - f(x)} .
	\end{align*}
\end{exer}

%------------------------------------------------
\begin{exer}
(5 pts)
Prove that if $f: D \ra \bR$ has a limit at $x_0$, then $|f|(x) := |f(x)|$ has a limit at $x_0$.
\end{exer}

%-------------------------------------------------
\begin{exer}
(10 pts)
Using the link between sequences and limits of functions, show the following.
	\begin{enumerate}[label=\textbf{\alph*)}]
	\item If $f(x) = x^n$ ($n \geq 0$), then $\lim_{x \ra x_0} f(x) = x_0^n$ for any $x_0 \in \bR$.
	\item If $x_0 \in [0, \infty )$, then $\lim_{x \ra x_0} \sqrt{x} = \sqrt{x_0}$.
\end{enumerate}	 
\end{exer}

\section{Bonus}
\begin{exer}
Assume that $f : \bR \ra \bR$ such that $f(x + y) = f(x) f(y)$ for all $x, y \in \bR$.
	\begin{enumerate}[label=\textbf{\alph*)}]
	\item Show that $f$ has a limit at every point of $\bR$.
	\item Show that either $\lim_{x \ra 0} f(x) = 1$ or $f(x) = 0$ for any $x \in \bR$.
	\end{enumerate}
\end{exer}


\end{document}