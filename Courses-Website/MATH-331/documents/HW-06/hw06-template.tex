\documentclass[12pt]{article}
\usepackage[utf8]{inputenc}

\usepackage{enumitem}
\usepackage[margin=2cm]{geometry}

\usepackage{amsmath, amsfonts, amssymb}
\usepackage{graphicx}
\usepackage{tikz}
\usepackage{pgfplots}
\usepackage{multicol}

\usepackage{comment}
\usepackage{url}

\usepackage{titlesec}
\titleformat{\section}[frame]
{\normalfont\scshape}
{\thesection}{8pt}{\centering}

\usepackage{array}

\pgfplotsset{compat=1.16}

%\usepackage[margin=1cm]{cloze}

\usepackage[thmmarks]{ntheorem}

% MATH commands
\newcommand{\bC}{\mathbb{C}}
\newcommand{\bR}{\mathbb{R}}
\newcommand{\bN}{\mathbb{N}}
\newcommand{\bZ}{\mathbb{Z}}
\newcommand{\bT}{\mathbb{T}}
\newcommand{\bD}{\mathbb{D}}
\newcommand{\bQ}{\mathbb{Q}}

\newcommand{\cL}{\mathcal{L}}
\newcommand{\cM}{\mathcal{M}}
\newcommand{\cP}{\mathcal{P}}
\newcommand{\cH}{\mathcal{H}}
\newcommand{\cB}{\mathcal{B}}
\newcommand{\cK}{\mathcal{K}}
\newcommand{\cJ}{\mathcal{J}}
\newcommand{\cU}{\mathcal{U}}
\newcommand{\cO}{\mathcal{O}}
\newcommand{\cA}{\mathcal{A}}
\newcommand{\cC}{\mathcal{C}}

\newcommand{\fK}{\mathfrak{K}}
\newcommand{\fM}{\mathfrak{M}}

\newcommand{\ga}{\left\langle}
\newcommand{\da}{\right\rangle}
\newcommand{\oa}{\left\lbrace}
\newcommand{\fa}{\right\rbrace}
\newcommand{\oc}{\left[}
\newcommand{\fc}{\right]}
\newcommand{\op}{\left(}
\newcommand{\fp}{\right)}

\newcommand{\ra}{\rightarrow}
\newcommand{\Ra}{\Rightarrow}

\renewcommand{\Re}{\mathrm{Re}\,}
\renewcommand{\Im}{\mathrm{Im}\,}
\newcommand{\Arg}{\mathrm{Arg}\,}
\newcommand{\Arctan}{\mathrm{Arctan}\,}
\newcommand{\sech}{\mathrm{sech}\,}
\newcommand{\csch}{\mathrm{csch}\,}
\newcommand{\Log}{\mathrm{Log}\,}
\newcommand{\cis}{\mathrm{cis}\,}

\newcommand{\ran}{\mathrm{ran}\,}
\newcommand{\bi}{\mathbf{i}}
\newcommand{\Sp}{\mathrm{span}\,}
\newcommand{\Inv}{\mathrm{Inv}\,}
\newcommand\smallO{
  \mathchoice
    {{\scriptstyle\mathcal{O}}}% \displaystyle
    {{\scriptstyle\mathcal{O}}}% \textstyle
    {{\scriptscriptstyle\mathcal{O}}}% \scriptstyle
    {\scalebox{.7}{$\scriptscriptstyle\mathcal{O}$}}%\scriptscriptstyle
  }
\newcommand{\HOL}{\mathrm{Hol}}
\newcommand{\cl}{\mathrm{clos}}
\newcommand{\ve}{\varepsilon}

\tikzstyle{myboxT} = [draw=black, fill=black!0,line width = 1pt,
    rectangle, rounded corners = 0pt, inner sep=8pt, inner ysep=8pt]
    
\newcommand{\MyC}[1]{\begin{tikzpicture}
\node (boxIntro) at (0,0) {};
\node [myboxT](Intro) at (boxIntro){%
	\begin{minipage}{0.9\textwidth}
	#1
	\end{minipage}};
\end{tikzpicture}}

%%%%  Environnement exer et solutionnaire
{\theorembodyfont{}
\theoremstyle{plain}
\theoremseparator{\textbf{.}}
\theoremsymbol{}
\newtheorem{exer}{\textbf{Exercise}}}

{\theorembodyfont{\color{black}}
\theoremstyle{plain}
\theoremseparator{\textbf{:}}
\theoremsymbol{$\square$}
\newtheorem*{sol}{\textbf{Solution}}}


\renewcommand*{\theexer}{\arabic{exer}}
\renewcommand*{\thesol}{\arabic{sol}}


\newcommand{\headHW}[4]{%
	\noindent \hrulefill \\
	MATH-#1 #2 \\
	#3 #4
	
}

\begin{document}

%%%%%%%%%%%%%%%%%%%%%%%%%%%%%%%%%%%%%%%5
%%%%%%%%%% HEADING HERE %%%%%%%%%%%%%%%%
%%%%%%%%%%%%%%%%%%%%%%%%%%%%%%%%%%%%%%%%%%
	\noindent \hrulefill \\
	MATH-331 Introduction to Real Analysis \hfill YOUR FULL NAME\\
	Homework 05 \hfill Fall 2021\\\vspace*{-0.7cm}
	
%%%%%%%%%%%%%%%%%%%%%%%%%%%%%%%%%%%%%%%%%55
%%%%%%%%%%%%%%%%%%%%%%%%%%%%%%%%%%%%%%%%
	
	\noindent\hrulefill
	
	\noindent Due date: November, 22${}^{\text{th}}$ 1:20pm \hfill Total: \hspace{0.3cm}/65.
	
\vspace*{0.5cm}

	\bgroup \renewcommand{\arraystretch}{1.5}
\begin{table}[h]
\centering
\begin{tabular}{|m{1.5cm}|>{\centering\arraybackslash}p{0.75cm}|>{\centering\arraybackslash}p{0.75cm}|>{\centering\arraybackslash}p{0.75cm}|>{\centering\arraybackslash}p{0.75cm}|>{\centering\arraybackslash}p{0.75cm}|>{\centering\arraybackslash}p{0.75cm}|>{\centering\arraybackslash}p{0.75cm}|>{\centering\arraybackslash}p{0.75cm}|>{\centering\arraybackslash}p{0.75cm}|>{\centering\arraybackslash}p{0.75cm}|}
\hline
Exercise & 1 (10) & 2 (10) & 3 (5) & 4 (5) & 5 (5) & 6 (10) & 7 (5) & 8 (5) & 9 (5) & 10 (5) \\
\hline
Score & & & & & & & & & &  \\\hline
\end{tabular}
\caption{Scores for each exercises}
\end{table}
\egroup
	
\vspace*{0.5cm}

{\bf Instructions:} You must answer all the questions below and send your solution by email (to \url{parisepo@hawaii.edu}). If you decide to not use {\LaTeX} to hand out your solutions, please be sure that after you scan your copy, it is clear and readable. Make sure that you attached a copy of the homework assignment to your homework. 

\noindent If you choose to use {\LaTeX}, you can use the template available on the course website.

\noindent No late homework will be accepted. No format other than PDF will be accepted. Name your file as indicated in the syllabus.

\section{Writing problems}
For each of the following problems, you will be asked to write a clear and detailed proof. You will have the chance to rewrite your solution in your semester project after receiving feedback from me.

%% ---------------------------------------------
\begin{exer}
(10 pts)
\begin{enumerate}[label=\textbf{\alph*)}]
\item Fix any $\delta > 0$ and let $[a, b]$ be an interval with $a < b$. Find a tagged partition $\cP$ of $[a, b]$ such that $\Vert \cP \Vert < \delta$.
\item Suppose that $f$ is Riemann integrable. Show that in the definition of the Riemann integral, the number $L$ is unique. [Remark: This is why we gave it the name $\int_a^b f$.]
\end{enumerate}
\end{exer}
\begin{sol}

\end{sol}

%-----------------------------------------------
\begin{exer}
(10 pts)
Suppose that $f$ and $g$ are Riemann integrable on the interval $[a, b]$.
	\begin{enumerate}[label=\textbf{\alph*)}]
	\item Show that $\int_a^b (f + g) = \int_a^b f + \int_a^b g$.
	\item Show that if $f(x) \leq g(x)$ for any $x \in [a, b]$, then $\int_a^b f \leq \int_a^b g$.
	\end{enumerate}
\end{exer}
\begin{sol}

\end{sol}

%-----------------------------------------------
\begin{exer}
(5 pts)
Let $f : [a, b] \ra \bR$ be Riemann integrable on $[a, b]$ and suppose that $|f(x)| \leq M$ for all $x \in [a, b]$. Show that $\int_a^b f \leq M (b - a)$.
\end{exer}
\begin{sol}

\end{sol}

%-----------------------------------------------
\begin{exer}
(5 pts)
Suppose that $f$ is Riemann integrable on $[a, b]$. Let $( \cP_n )_{n =1}^\infty$ be a sequence of tagged partitions of $[a, b]$ such that the sequence $\lim_{n \ra \infty} \Vert \cP_n \Vert = 0$. Prove that the sequence $( S (f , \cP_n ))_{n = 1}^\infty$ converges to $\int_a^b f$.
\end{exer}
\begin{sol}

\end{sol}


%-----------------------------------------------
\begin{exer}
(5 pts)
Let $f : [a, b] \ra \bR$ be a bounded function. Suppose that $f$ is Riemann integrable on $[a, c]$ for any $c \in (a, b)$. Show that $f$ is Riemann integrable on $[a, b]$. [Hint: Use the Cauchy criterion for integrals.]
\end{exer}
\begin{sol}

\end{sol}

\section{Homework problems}
Answer all the questions below. Make sure to show your work.


\begin{exer}
(10pts)
\begin{enumerate}[label=\textbf{\alph*)}]
\item Define the function $f : [a, b] \ra \bR$ by $f(x) = x$ for every $x \in [a, b]$ where $k \in \bR$ is a fixed constant. Show that $f$ is Riemann integrable on $[a, b]$ and that $\int_a^b x \, dx = (b^2 - a^2)/2$.
\item Define the function $f : [a, b] \ra \bR$ by $f(x) = \sin x$ for every $x \in [0, \pi/2]$. Show that $f$ is Riemann integrable on $[0, \pi/2 ]$. What is the value of $\int_0^{\pi/2} f$?
\end{enumerate}
\end{exer}
\begin{sol}

\end{sol}

%------------------------------------------------
\begin{exer}
(5 pts)
Show that the function $f : [0, 1] \ra \bR$ defined by
	\begin{align*}
	f(x) := \begin{cases}
	1 & \text{, if } 0 \leq x < 1/2 \\
	0 & \text{, if } 1/2 \leq x \leq 1 
	\end{cases}
	\end{align*}
is Riemann integrable on $[0, 1]$.
\end{exer}
\begin{sol}

\end{sol}

%-------------------------------------------------
\begin{exer}
(5 pts)
Let $f : [0, 1] \ra \bR$ be defined by $f(x) = 1$ if $x = 1/n$ where $n \in \bN$, and by $f(x) = 0$ if $x \neq 1/n$, $n \in \bN$. Show that $f$ is Riemann integrable on $[0, 1]$.
\end{exer}
\begin{sol}

\end{sol}

%------------------------------------------------
\begin{exer}
(5 pts)
Show that the function $f : [0, 1] \ra \bR$ defined by $f(x) = 0$ if $x \neq 0$ and $f(x) = 4$ if $x = 0$ is Riemann integrable on $[0, 1]$.
\end{exer}
\begin{sol}
	
\end{sol}

%------------------------------------------------
\begin{exer}
(5 pts)
Let $\cP$ be the following tagged partition of $[-1, 2]$:
	\begin{align*}
	\cP &:= \{ (-9, [-1, -.8]) , (-.7, [-.8, -.3]), (-.1, [-.3, 0]), (.2, [0,0.2]), (.2, [.2, .4]), (.8, [.4, 1]), \\
	& \phantom{:= \{} (1.42, [1, 1.5]), (1.9, [1.5, 2]) \} .
	\end{align*}
Find another partition $\cP_0$ such that $\Vert \cP_0 \Vert \leq \Vert \cP \Vert/3$.
\end{exer}
\begin{sol}

\end{sol}


\end{document}