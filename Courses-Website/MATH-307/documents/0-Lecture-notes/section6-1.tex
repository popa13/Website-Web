\documentclass[12pt,a4paper]{article}
\usepackage[utf8]{inputenc}
\usepackage[english]{babel}

\usepackage{amsmath}
\usepackage{amsfonts}
\usepackage{amssymb}

\usepackage{graphicx}
\usepackage{lmodern}
\usepackage{tikz}
\usepackage{titlesec}
\usepackage{environ}
\usepackage{xcolor}
\usepackage{fancyhdr}
\usepackage[colorlinks = true, linkcolor = black]{hyperref}
\usepackage{xparse}
\usepackage{enumerate}

\usepackage[left=2cm,right=2cm,top=2cm,bottom=2cm]{geometry}
\usepackage{multicol}
\usepackage[indent=0pt]{parskip}

\newcommand{\spaceP}{\vspace*{0.5cm}}
\newcommand{\Span}{\mathrm{Span}\,}
\newcommand{\range}{\mathrm{range}\,}

%% Redefining sections
\newcommand{\sectionformat}[1]{%
    \begin{tikzpicture}[baseline=(title.base)]
        \node[rectangle, draw] (title) {#1};
    \end{tikzpicture}
    
    \noindent\hrulefill
}

% default values copied from titlesec documentation page 23
% parameters of \titleformat command are explained on page 4
\titleformat%
    {\section}% <command> is the sectioning command to be redefined, i. e., \part, \chapter, \section, \subsection, \subsubsection, \paragraph or \subparagraph.
    {\normalfont\large\scshape}% <format>
    {}% <label> the number
    {0em}% <sep> length. horizontal separation between label and title body
    {\centering\sectionformat}% code preceding the title body  (title body is taken as argument)

%% Set counters for sections to none
\setcounter{secnumdepth}{0}

%% Set the footer/headers
\pagestyle{fancy}
\fancyhf{}
\renewcommand{\headrulewidth}{0pt}
\renewcommand{\footrulewidth}{2pt}
\lfoot{P.-O. Paris{\'e}}
\cfoot{MATH 307}
\rfoot{Page \thepage}

%% Defining example environment
\newcounter{example}[section]
\NewEnviron{example}%
	{%
	\noindent\refstepcounter{example}\fcolorbox{gray!40}{gray!40}{\textsc{\textcolor{red}{Example~\theexample.}}}%
	%\fcolorbox{black}{white}%
		{  %\parbox{0.95\textwidth}%
			{
			\BODY
			}%
		}%
	}

% Theorem environment
\NewEnviron{theorem}%
	{%
	\noindent\refstepcounter{example}\fcolorbox{gray!40}{gray!40}{\textsc{\textcolor{blue}{Theorem~\theexample.}}}%
	%\fcolorbox{black}{white}%
		{  %\parbox{0.95\textwidth}%
			{
			\BODY
			}%
		}%
	}

%% Commands for matrix of dimensions m x n
%\newcommand{\matMN}[1]{%
%	\begin{bmatrix}
%	#1_{11} & #1_{12} & #1_{13} & \cdots & #1_{1n} \\
%	#1_{21} & #1_{22} & #1_{23} & \cdots & #1_{2n} \\
%	#1_{31} & #1_{32} & #1_{33} & \cdots & #1_{3n} \\
%	\vdots & \vdots & \vdots & \ddots & \vdots \\
%	#1_{m1} & #1_{m2} & #1_{m3} & \cdots & #1_{mn}
%	\end{bmatrix}		
%	}
	
\NewDocumentCommand{\matMN}{mg}{%
	\IfNoValueTF{#2}
		{%
		  \begin{bmatrix}
		   #1_{11} & #1_{12} & #1_{13} & \cdots & #1_{1n} \\
	       #1_{21} & #1_{22} & #1_{23} & \cdots & #1_{2n} \\
		   #1_{31} & #1_{32} & #1_{33} & \cdots & #1_{3n} \\
		   \vdots & \vdots & \vdots & \ddots & \vdots \\
		   #1_{m1} & #1_{m2} & #1_{m3} & \cdots & #1_{mn}
		   \end{bmatrix}	%
		}
		{%
		   \begin{bmatrix}
		   #1_{11} + #2_{11} & #1_{12} + #2_{12} & \cdots & #1_{1n} + #2_{2n} \\
		   #1_{21} + #2_{21} & #1_{22} + #2_{22} & \cdots & #1_{2n} + #2_{2n} \\
		   \vdots & \vdots & \vdots & \ddots & \vdots \\
		   #1_{m1} + #2_{m1} & #1_{m2} + #2_{m2} & \cdots & #1_{mn} + #2_{mn}
		   \end{bmatrix}
		}%
}

%%%%
\begin{document}
\thispagestyle{empty}

\begin{center}
\vspace*{2.5cm}

{\Huge \textsc{Math 307}}

\vspace*{2cm}

{\LARGE \textsc{Chapter 6}} 

\vspace*{0.75cm}

\noindent\textsc{Section 6.1: The Theory of Systems of Linear Differential Equations}

\vspace*{0.75cm}

\tableofcontents

\vfill

\noindent \textsc{Created by: Pierre-Olivier Paris{\'e}} \\
\textsc{Summer 2022}
\end{center}

\newpage

\section{Mixing Problems}

\begin{example}
Consider two tanks each with volume $100$ gallons. The two tanks are connected together by two pipes. The first tank initially contains a well-mixed solution of $5$lb salt in $50$gal water. The second tank initially contains $100$ gal salt-free water. 

A pipe from tank 1 to tank 2 allows the solution in tank 1 to enter tank 2 at a rate of $5$ gal/min. A second pipe from tank 2 to tank 1 allows the solution from tank 2 to enter tank 1 at a rate of $5$ gal/min.

Assume that the salt mixture in each tank is well-stirred. Find a model describing the quantity of salt in each tank.
\end{example}

\newpage

\section{Definitions}

\subsection{System of ODEs}
A \textbf{system of $n$ first order linear differential equations} (system of $n$ ODEs for short) is a vector-equation:
	\begin{align*}
	Y' = AY + G
	\end{align*}
where
	\begin{itemize}
	\item $Y$ is an $n \times 1$ vector of unknown functions:
		\begin{align*}
		Y (x) = \begin{bmatrix}
		y_1 (x) \\ y_2 (x) \\ \vdots \\ y_n (x)
		\end{bmatrix} .
		\end{align*}
	\item $Y'$ is the $n \times 1$ vector of derivatives of the unknown functions:
		\begin{align*}
		Y' (x) = \begin{bmatrix}
		y_1'(x) \\ y_2' (x) \\ \vdots \\ y_n'(x)
		\end{bmatrix} .
		\end{align*}
	\item $A$ is an $n \times n$ matrix of functions:
		\begin{align*}
		A = \begin{bmatrix}
		a_{11}(x) & a_{12} (x) & \cdots & a_{1n} (x) \\
		a_{21} (x) & a_{22} (x) & \cdots & a_{2n} (x) \\
		\vdots & \vdots & \ddots & \vdots \\
		a_{n1} (x) & a_{n2} (x) & \cdots & a_{nn} (x)
		\end{bmatrix} .
		\end{align*}
	\item $G$ is an $n \times 1$ column vector of functions:
		\begin{align*}
		G (x) = \begin{bmatrix}
		g_1(x) \\ g_2 (x) \\ \vdots \\ g_n (x)
		\end{bmatrix} .
		\end{align*}
	\end{itemize}
	
If we add the additional conditions $Y (x_0) = B$ for some real number $x_0$ and an $n \times 1$ column vector $B$, the system of ODEs is called an \textbf{initial value problem}.
	
\subsection{Homogeneous and Non-homogeneous}
	\begin{itemize}
	\item If $G (x) = 0$ for every $x$, the system of ODEs is called \textbf{homogeneous}.
	\item if $G(x)$ is not zero, then the system of ODEs is called \textbf{non-homogeneous}.
	\end{itemize}
	
\newpage

\begin{example}
Consider the following system of ODEs:
	\begin{align*}
	Y' = \begin{bmatrix}
	4 & -1 \\ 2 & 1
	\end{bmatrix} Y .
	\end{align*}
	\begin{enumerate}
	\item Is this a homogeneous or non-homogeneous system of ODEs?
	\item Show that
		\begin{align*}
		Y (x) = \begin{bmatrix}
		e^{2x} + e^{3x} \\ 2e^{2x} + e^{3x}
		\end{bmatrix}
		\end{align*}
	is a solution to the system.
	\end{enumerate}
\end{example}

\vfill

\begin{example}
Consider the following initial value problem:
	\begin{align*}
	Y' = \begin{bmatrix}
	4 & -1 \\ 2 & 1
	\end{bmatrix} Y  \quad \text{ and } \quad Y(0) = \begin{bmatrix}
	3 \\ 5
	\end{bmatrix}
	\end{align*}
Show that
	\begin{align*}
	Y (x) = \begin{bmatrix}
	2e^{2x} + e^{3x} \\ 4e^{2x} + e^{3x}
	\end{bmatrix}
	\end{align*}
is a solution to the initial value problem.
\end{example}

\vfill

\newpage
	
\section{Do Solutions to a System of ODEs Exist?}

	\subsection{Existence and Uniqueness Theorem}
	Consider the initial value problem
		\begin{align}
		Y' = AY + G \quad \text{ and } \quad Y(x_0) = B . \tag{$\star$} \label{Eq:IVPSysODEs}
		\end{align}
	If all the entries $a_{ij} (x)$ of $A$ and all the entries $g_i (x)$ of $G$ are continuous functions, then the initial value problem \eqref{Eq:IVPSysODEs} has a unique solution.
	
	\subsection{Solutions as a Subspace}
	\begin{example}
	Consider the following system of ODEs:
		\begin{align*}
		Y' = \begin{bmatrix}
	4 & -1 \\ 2 & 1
	\end{bmatrix} Y .
		\end{align*}
	If the general solution to the system is
		\begin{align*}
		Y (x) = \begin{bmatrix}
		c_1 e^{2x} + c_2 e^{3x} \\ 2c_1 e^{2x} + c_2 e^{3x}
		\end{bmatrix},
		\end{align*}
	describe the structure of the set of solutions.
	\end{example}
	
	\vfill
	
	\underline{Fact}: The set of solutions to a homogeneous system of $n$ ODEs $Y' = A(x) Y$ form a vector space of dimension $n$.
	
	\newpage
	
	\subsection{Nomenclature}
		\begin{itemize}
		\item A set of $n$ linearly independent solutions $Y_1$, $Y_2$, $\ldots$, $Y_n$ to a homogeneous system of $n$ ODEs is called a \textbf{fundamental set of solutions}.
		\item A \textbf{general solution}, denoted by $Y_H$, to a homogeneous system of $n$ ODEs with fundamental set of solutions $Y_1$, $Y_2$, $\ldots$, $Y_n$ is a linear combination of $Y_1$, $Y_2$, $\ldots$, $Y_n$, that is
			\begin{align*}
			Y_H = c_1 Y_1 + c_2 Y_2 + \cdots + c_n Y_n .
			\end{align*}
		\item The \textbf{matrix of fundamental solutions}, denoted by $M$, is the matrix $M$ form by the vector functions $Y_1$, $Y_2$, $\ldots$, $Y_n$ in the fundamental set of solutions:
			\begin{align*}
			M = \begin{bmatrix}
			Y_1 & Y_2 & \cdots & Y_n
			\end{bmatrix} = \begin{bmatrix}
			y_{11} & y_{12} & \cdots & y_{1n} \\
			y_{21} & y_{22} & \cdots & y_{2n} \\
			\vdots & \vdots & \ddots & \vdots \\
			y_{n1} & y_{n2} & \cdots & y_{nn}
			\end{bmatrix} .
			\end{align*}
		\end{itemize}
		
	\vspace*{3cm}
	
	\subsection{Non-homogeneous Systems}
	Solutions to non-homogeneous systems and homogeneous system are related by one thing:
		\begin{itemize}
		\item A \textbf{particular solution} to a system $Y' = AY + G$, denoted by $Y_P$, is a specific solution to the system.
		\end{itemize}
		
	Therefore, every solution $Y$ to the system $Y' = AY + G$ has the form
		\begin{align*}
		Y = Y_H + Y_P = c_1 Y_1 + c_2 Y_2 + \cdots + c_n Y_n + Y_P = MC + Y_P
		\end{align*}
	where
		\begin{itemize}
		\item $Y_H$ is the general solution to the system $Y' = AY$.
		\item $Y_P$ is a particular solution to the system $Y' = AY + B$.
		\end{itemize}
		
	
	\newpage
	
	\section{Wronkian}
	
	\subsection{Definition}
	Given $n$ column vector functions
		\begin{align*}
		Y_1 (x) = \begin{bmatrix}
		y_{11} (x) \\ y_{21} (x) \\ \vdots \\ y_{n1} (x)
		\end{bmatrix} , \quad Y_2 (x) = \begin{bmatrix}
		y_{21} (x) \\ y_{22} (x) \\ \vdots \\ y_{n2} (x)
		\end{bmatrix} , \quad \cdots , \quad
		Y_n (x) = \begin{bmatrix}
		y_{n1} (x) \\ y_{n2} (x) \\ \vdots \\ y_{nn} (x)
		\end{bmatrix}
		\end{align*}
	then the \textbf{Wronkian} of $Y_1$, $Y_2$, $\ldots$, $Y_n$ is defined as
		\begin{align*}
		w (Y_1 (x), Y_2 (x), \cdots , Y_n (x)) := \begin{vmatrix}
		y_{11} (x) & y_{12} (x) & \cdots & y_{1n} (x) \\
		y_{21} (x) & y_{22} (x) & \cdots & y_{2n} (x) \\
		\vdots & \vdots & \ddots & \vdots \\
		y_{n1} (x) & y_{n2} (x) & \cdots & y_{nn} (x) 
		\end{vmatrix} .
		\end{align*}
		
	\begin{example}\label{Ex:WronkianCalc}
	Let $Y_1$ and $Y_2$ be the vector functions
		\begin{align*}
		Y_1 (x) = \begin{bmatrix}
		e^{2x} \\ 2e^{2x}
		\end{bmatrix} \quad \text{ and } \quad
		Y_2 (x) = \begin{bmatrix}
		e^{3x} \\ e^{3x}
		\end{bmatrix} .
		\end{align*}
	Compute $w (Y_1 (x), Y_2 (x))$.
	\end{example}
	
	
	\newpage
	
	\subsection{Linear Independence of Vector Functions}
	
	\begin{example}
	Show that the vector functions in Example \ref{Ex:WronkianCalc} are linearly independent.
	\end{example}
	
	\newpage
	
	\phantom{2}
	
	\vfill
	
	\underline{Main Important Fact:}
	
	Given a list $Y_1$, $Y_2$, $\cdots$, $Y_n$ of vector functions, if $w (Y_1 (x) , Y_2 (x) , \ldots , Y_n (x)) \neq 0$ for some $x$, then $Y_1$, $Y_2$, $\cdots$, $Y_n$ are linearly independent.
	
	\vspace*{14pt}
	
	\underline{Other Facts}:
		\begin{itemize}
		\item If $Y_1$, $Y_2$, $\ldots$, $Y_n$ are linearly dependent, then $w (Y_1 (x) , Y_2 (x) , \ldots , Y_n (x)) = 0$ for any $x$.
		\item If $Y_1$, $Y_2$, $\ldots$, $Y_n$ are solutions to $Y' = AY$ and if $w (Y_1(x) , Y_2(x) , \ldots , Y_n (x)) = 0$ for some $x$, then $Y_1$, $Y_2$, $\ldots$, $Y_n$ are linearly dependent.
		\item If $Y_1$, $Y_2$, $\ldots$, $Y_n$ is a fundamental set of solutions to $Y' = AY$, then 
		$$
		w(Y_1(x) , Y_2 (x) , \ldots , Y_n (x) ) \neq 0
		$$
	for every $x$.
		\end{itemize}
		
\newpage

\section{Solving Diagonal Systems}
Our investigations in the next chapter will focus mainly on system of $n$ ODEs with constant coefficients. This means:
	\begin{center}
	The entries of the matrix $A$ in the equation $Y' = AY + G$ are constants.
	\end{center}
	
We begin with the case of a diagonal matrix $A$.

\vspace*{14pt}

	\begin{example}
	Solve the system
		\begin{align*}
		Y' = \begin{bmatrix}
		3 & 0 \\ 0 & -1
		\end{bmatrix} Y .
		\end{align*}
	\end{example}
	
\newpage

The general solution to a homogeneous system $Y' = AX$ where $A$ is a diagonal matrix
	\begin{align*}
	A = \begin{bmatrix}
	d_1 & 0 & \cdots & 0 \\
	0 & d_2 & \cdots & 0 \\
	\vdots & \vdots & \ddots & \vdots \\
	0 & 0 & \cdots & d_n
	\end{bmatrix}
	\end{align*} 
is given by
	\begin{align*}
	Y_H = \begin{bmatrix}
	e^{d_1x} & 0 & \cdots & 0 \\
	0 & e^{d_2x} & \cdots & 0 \\
	\vdots & \vdots & \ddots & \vdots \\
	0 & 0 & \cdots & e^{d_n x}
	\end{bmatrix} \begin{bmatrix}
	c_1 \\ c_2 \\ \vdots \\ c_n
	\end{bmatrix} = \begin{bmatrix}
	c_1 e^{d_1 x} \\ c_2 e^{d_2x} \\ \vdots \\ c_n e^{d_n x}
	\end{bmatrix} .
	\end{align*}
	
	\begin{example}
	Solve the initial value problem
		\begin{align*}
		Y' = \begin{bmatrix} 
		-3 & 0 & 0 & 0 \\
		0 & -2 & 0 & 0 \\
		0 & 0 & 2 & 0 \\
		0 & 0 & 0 & 5 
		\end{bmatrix} Y \quad \text{ and } \quad 
		Y (0) = \begin{bmatrix}
		2 \\ 1 \\ -1 \\ 0
		\end{bmatrix} .
		\end{align*}
	\end{example}
	
\end{document}