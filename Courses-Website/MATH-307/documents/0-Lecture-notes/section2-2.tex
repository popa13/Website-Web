\documentclass[12pt,a4paper]{article}
\usepackage[utf8]{inputenc}
\usepackage[english]{babel}

\usepackage{amsmath}
\usepackage{amsfonts}
\usepackage{amssymb}

\usepackage{graphicx}
\usepackage{lmodern}
\usepackage{tikz}
\usepackage{titlesec}
\usepackage{environ}
\usepackage{xcolor}
\usepackage{fancyhdr}
\usepackage[colorlinks = true, linkcolor = black]{hyperref}
\usepackage{xparse}
\usepackage{enumerate}

\usepackage[left=2cm,right=2cm,top=2cm,bottom=2cm]{geometry}
\usepackage{multicol}
\usepackage[indent=0pt]{parskip}

\newcommand{\spaceP}{\vspace*{0.5cm}}
\newcommand{\Span}{\mathrm{Span}\,}

%% Redefining sections
\newcommand{\sectionformat}[1]{%
    \begin{tikzpicture}[baseline=(title.base)]
        \node[rectangle, draw] (title) {#1};
    \end{tikzpicture}
    
    \noindent\hrulefill
}

% default values copied from titlesec documentation page 23
% parameters of \titleformat command are explained on page 4
\titleformat%
    {\section}% <command> is the sectioning command to be redefined, i. e., \part, \chapter, \section, \subsection, \subsubsection, \paragraph or \subparagraph.
    {\normalfont\large\scshape}% <format>
    {}% <label> the number
    {0em}% <sep> length. horizontal separation between label and title body
    {\centering\sectionformat}% code preceding the title body  (title body is taken as argument)

%% Set counters for sections to none
\setcounter{secnumdepth}{0}

%% Set the footer/headers
\pagestyle{fancy}
\fancyhf{}
\renewcommand{\headrulewidth}{0pt}
\renewcommand{\footrulewidth}{2pt}
\lfoot{P.-O. Paris{\'e}}
\cfoot{MATH 307}
\rfoot{Page \thepage}

%% Defining example environment
\newcounter{example}[section]
\NewEnviron{example}%
	{%
	\noindent\refstepcounter{example}\fcolorbox{gray!40}{gray!40}{\textsc{\textcolor{red}{Example~\theexample.}}}%
	%\fcolorbox{black}{white}%
		{  %\parbox{0.95\textwidth}%
			{
			\BODY
			}%
		}%
	}

% Theorem environment
\NewEnviron{theorem}%
	{%
	\noindent\refstepcounter{example}\fcolorbox{gray!40}{gray!40}{\textsc{\textcolor{blue}{Theorem~\theexample.}}}%
	%\fcolorbox{black}{white}%
		{  %\parbox{0.95\textwidth}%
			{
			\BODY
			}%
		}%
	}

%% Commands for matrix of dimensions m x n
%\newcommand{\matMN}[1]{%
%	\begin{bmatrix}
%	#1_{11} & #1_{12} & #1_{13} & \cdots & #1_{1n} \\
%	#1_{21} & #1_{22} & #1_{23} & \cdots & #1_{2n} \\
%	#1_{31} & #1_{32} & #1_{33} & \cdots & #1_{3n} \\
%	\vdots & \vdots & \vdots & \ddots & \vdots \\
%	#1_{m1} & #1_{m2} & #1_{m3} & \cdots & #1_{mn}
%	\end{bmatrix}		
%	}
	
\NewDocumentCommand{\matMN}{mg}{%
	\IfNoValueTF{#2}
		{%
		  \begin{bmatrix}
		   #1_{11} & #1_{12} & #1_{13} & \cdots & #1_{1n} \\
	       #1_{21} & #1_{22} & #1_{23} & \cdots & #1_{2n} \\
		   #1_{31} & #1_{32} & #1_{33} & \cdots & #1_{3n} \\
		   \vdots & \vdots & \vdots & \ddots & \vdots \\
		   #1_{m1} & #1_{m2} & #1_{m3} & \cdots & #1_{mn}
		   \end{bmatrix}	%
		}
		{%
		   \begin{bmatrix}
		   #1_{11} + #2_{11} & #1_{12} + #2_{12} & \cdots & #1_{1n} + #2_{2n} \\
		   #1_{21} + #2_{21} & #1_{22} + #2_{22} & \cdots & #1_{2n} + #2_{2n} \\
		   \vdots & \vdots & \vdots & \ddots & \vdots \\
		   #1_{m1} + #2_{m1} & #1_{m2} + #2_{m2} & \cdots & #1_{mn} + #2_{mn}
		   \end{bmatrix}
		}%
}

%%%%
\begin{document}
\thispagestyle{empty}

\begin{center}
\vspace*{2.5cm}

{\Huge \textsc{Math 307}}

\vspace*{2cm}

{\LARGE \textsc{Chapter 2}} 

\vspace*{0.75cm}

\noindent\textsc{Section 2.2: Subspaces and Spanning Sets}

\vspace*{0.75cm}

\tableofcontents

\vfill

\noindent \textsc{Created by: Pierre-Olivier Paris{\'e}} \\
\textsc{Summer 2022}
\end{center}

\newpage

\section{What Is a Subspace}

	\subsection{Definition}

	In loose terms, a \textbf{subspace} is simply a vector space inside another vector space. Precisely, a subspace is a subset $W$ of another vector space $V$ such that $W$ is itself a vector space under the same addition and scalar multiplication operations of $V$ restricted to $W$.
	
	The next result tells us that we only need to verify if the operations are closed.
	
	\vspace*{18pt}
	
	\begin{theorem}
	Let $W$ be a nonempty subset of a vector space $V$. Then $W$ is a subspace of $V$ if and only if for all vectors $u$ and $w$ in $W$ and for all scalar $c$, we have 
		\begin{itemize}
		\item $u + w$ is in $W$;
		\item $c u$ is in $W$.
		\end{itemize}
	\end{theorem}
	
	\vspace*{18pt}
	
	\begin{example}
	Let $W$ be the set of all column vectors of the form
		\begin{align*}
		\begin{bmatrix}
		x \\ y \\ 0
		\end{bmatrix} .
		\end{align*}
	Show that $W$ is a subspace of the vector space of all column vectors.
	\end{example}
	
	\vfill
	
	\begin{example}
	Do the set of vectors of the form
		\begin{align*}
		\begin{bmatrix}
		x \\ 1
		\end{bmatrix}
		\end{align*}
	forms a subspace of $\mathbb{R}^2$?
	\end{example}
	
	\vfill
	
	\newpage
	
	\begin{example}
	Do the set of vectors of the form
		\begin{align*}
		\begin{bmatrix}
		x \\ y \\ x - 2y
		\end{bmatrix}
		\end{align*}
	forms a subspace of $\mathbb{R}^3$?
	\end{example}
	

	\newpage
	
	%\subsection{Subspaces Generated By a Set of Solutions}
	
	%\begin{theorem}
	%If $A$ is an $m \times n$ matrix such that the system of linear equations $AX = O_{m \times 1}$ has at least one solution, then the set of solutions to $AX = O_{m \times 1}$ is a subspace of $\mathbb{R}^n$.
	%\end{theorem}
	
	%\vspace*{18pt}
	
	%\noindent It is instructive to verify the last theorem. We will prove it.
	
	%\vspace*{8pt}
	
	%\noindent\underline{\textbf{Proof.}}
	
	%\newpage
	
	\subsection{Important Examples: Set of Polynomials}
	Let $n$ be a nonnegative integer and let $P_{n}$ denote the set of polynomials of degree less than or equal to $n$ on $(a, b)$; that is the set of expressions $p(x)$ of the form
		\begin{align*}
		p(x) = a_kx^k + a_{k-1}x^{k-1} + \cdots + a_1 x + a_0 
		\end{align*}
	for $k$ an integer such that $k \leq n$.
	
	\begin{example}
	Let $P_2$ denote the set of polynomials of degree less than or equal to $2$ on $(a, b )$; that is the set of expressions $p(x)$ of the form
		\begin{align*}
		p(x) = ax^2 + bx + c .
		\end{align*}
	Show that $P_2$ is a subspace of the vector space of functions $F (a, b)$.
	\end{example}
	
	\vfill
	
	\underline{Fact}:
	Let $P$ denote the set of all polynomials on $(a, b)$. This means $P$ is the set of expressions $p(x)$ of the form
		\begin{align*}
		p(x) = a_n x^n + a_{n-1}x^{n-1} + \cdots + a_1 x + a_0 .
		\end{align*}
	Show that $P$ is a subspace of the vector space of functions $F(a, b)$.
	
	
	%\begin{example}
	%The set of differentiable functions on $(a, b)$ denoted by $D (a, b)$ is a nonempty subspace of the vector space $F(a, b)$. This comes from Calculus I because we know that
	%	\begin{itemize}
	%	\item The sum of two differentiable functions is differentiable;
	%	\item A constant multiple of a differentiable function is differentiable.
	%	\end{itemize}
	%\end{example}
	
	%\vspace*{18pt}
	
	%\begin{example}
	%For each positive integer $n$, denote by $C^n (a, b)$ the set of all functions that have continuous $n$-th derivative on $(a, b)$. Then,
	%	\begin{itemize}
	%	\item $C^{n + 1} (a, b)$ is a subspace of $C^n (a, b)$;
	%	\item $C^n (a, b)$ is a subspace of $D (a, b)$.
	%	\item $C^n (a, b)$ is a subspace of $F (a, b)$.
	%	\end{itemize}
	%\end{example}
	
	\newpage
	
\section{What does Span mean?}

	\subsection{Linear Combinations}
	
	Given a bunch of vectors $v_1$, $v_2$, $\ldots$, $v_n$ in a vector space $V$, a \textbf{linear combination} of these vectors is
		\begin{align*}
		c_1 v_1 + c_2 v_2 + \cdots + c_n v_n
		\end{align*}
	for some scalars $c_1$, $c_2$, $\ldots$, $c_n$.
	
	\vspace*{18pt}
	
	\begin{example}
	Is the polynomial $v(x) = 2x^2 + x + 1$ a linear combination of the polynomials $v_1 (x) = x^2 + 1$, $v_2 (x) = x^2 - 1$, $v_3 (x) = x + 1$?
	\end{example}
	
	\newpage
	
	\subsection{Spanning set}
	
	The set of all linear combinations of vectors $v_1$, $v_2$, $\ldots$, $v_n$ of $V$ is called the \textbf{spanning set of} $\mathbf{v_1}$\textbf{,} $\mathbf{v_2}$\textbf{,} $\mathbf{\ldots}$\textbf{,} $\mathbf{v_n}$.
	
	\vspace*{12pt}
	
	\noindent The notation for the spanning set of the subspace of $V$ generated by the vectors $v_1$, $v_2$, $\ldots$, $v_n$ is
		\begin{align*}
		\Span \{ v_1 , v_2 , \ldots , v_n \} .
		\end{align*}
	
	\vspace*{16pt}
	
	\begin{example}
	Is the vector
		\begin{align*}
		\begin{bmatrix}
		2 \\ -5 \\ 1 \\ 10
		\end{bmatrix}
		\text{ in the }
		\Span \left\lbrace 
		\begin{bmatrix}
		1 \\ -1 \\ 2 \\ 3
		\end{bmatrix} , 
		\begin{bmatrix}
		1 \\ -2 \\ -1 \\ 2
		\end{bmatrix}
		, 
		\begin{bmatrix}
		-1 \\ 0 \\ 1 \\ 3
		\end{bmatrix}
		\right\rbrace \text{ ?}
		\end{align*}
	\end{example}
	
	\newpage
	
	\subsection{Spanning a whole vector space}
	
	We say that the vectors $v_1$, $v_2$, $\ldots$, $v_n$ of a vector space $V$ span $V$ if
		\begin{align*}
		\Span \{ v_1 , v_2 , \ldots , v_n \} = V .
		\end{align*}
	In other words, each vector in $V$ is a linear combination of the vectors $v_1$, $v_2$, $\ldots$, $v_n$.
	
	\vspace*{18pt}
	
	\begin{example}
	Do
		\begin{align*}
		v_1 = \begin{bmatrix}
		1 \\ -2
		\end{bmatrix} ,
		v_2 = \begin{bmatrix}
		2 \\ -4
		\end{bmatrix}
		\end{align*}
	span $\mathbb{R}^2$?
	\end{example}
	
	\newpage
	
	\begin{example}
	Let $v_1(x) = x^2 + x - 3$, $v_2 (x) = x - 5$, $v_3 (x) = 3$, and $v_4 (x) = x + 1$.
		\begin{enumerate}
		\item Do $v_1$, $v_2$, $v_3$ span $P_2$?
		\item Do $v_2$, $v_3$, $v_4$ span $P_1$?
		\item Do $v_1$, $v_2$, $v_3$ span $P_3$?
		\end{enumerate}
	\end{example}
	
	\newpage
	
	\phantom{2}

\end{document}