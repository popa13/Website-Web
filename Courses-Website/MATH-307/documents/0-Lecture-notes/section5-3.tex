\documentclass[12pt,a4paper]{article}
\usepackage[utf8]{inputenc}
\usepackage[english]{babel}

\usepackage{amsmath}
\usepackage{amsfonts}
\usepackage{amssymb}

\usepackage{graphicx}
\usepackage{lmodern}
\usepackage{tikz}
\usepackage{titlesec}
\usepackage{environ}
\usepackage{xcolor}
\usepackage{fancyhdr}
\usepackage[colorlinks = true, linkcolor = black]{hyperref}
\usepackage{xparse}
\usepackage{enumerate}

\usepackage[left=2cm,right=2cm,top=2cm,bottom=2cm]{geometry}
\usepackage{multicol}
\usepackage[indent=0pt]{parskip}

\newcommand{\spaceP}{\vspace*{0.5cm}}
\newcommand{\Span}{\mathrm{Span}\,}
\newcommand{\range}{\mathrm{range}\,}

%% Redefining sections
\newcommand{\sectionformat}[1]{%
    \begin{tikzpicture}[baseline=(title.base)]
        \node[rectangle, draw] (title) {#1};
    \end{tikzpicture}
    
    \noindent\hrulefill
}

% default values copied from titlesec documentation page 23
% parameters of \titleformat command are explained on page 4
\titleformat%
    {\section}% <command> is the sectioning command to be redefined, i. e., \part, \chapter, \section, \subsection, \subsubsection, \paragraph or \subparagraph.
    {\normalfont\large\scshape}% <format>
    {}% <label> the number
    {0em}% <sep> length. horizontal separation between label and title body
    {\centering\sectionformat}% code preceding the title body  (title body is taken as argument)

%% Set counters for sections to none
\setcounter{secnumdepth}{0}

%% Set the footer/headers
\pagestyle{fancy}
\fancyhf{}
\renewcommand{\headrulewidth}{0pt}
\renewcommand{\footrulewidth}{2pt}
\lfoot{P.-O. Paris{\'e}}
\cfoot{MATH 307}
\rfoot{Page \thepage}

%% Defining example environment
\newcounter{example}[section]
\NewEnviron{example}%
	{%
	\noindent\refstepcounter{example}\fcolorbox{gray!40}{gray!40}{\textsc{\textcolor{red}{Example~\theexample.}}}%
	%\fcolorbox{black}{white}%
		{  %\parbox{0.95\textwidth}%
			{
			\BODY
			}%
		}%
	}

% Theorem environment
\NewEnviron{theorem}%
	{%
	\noindent\refstepcounter{example}\fcolorbox{gray!40}{gray!40}{\textsc{\textcolor{blue}{Theorem~\theexample.}}}%
	%\fcolorbox{black}{white}%
		{  %\parbox{0.95\textwidth}%
			{
			\BODY
			}%
		}%
	}

%% Commands for matrix of dimensions m x n
%\newcommand{\matMN}[1]{%
%	\begin{bmatrix}
%	#1_{11} & #1_{12} & #1_{13} & \cdots & #1_{1n} \\
%	#1_{21} & #1_{22} & #1_{23} & \cdots & #1_{2n} \\
%	#1_{31} & #1_{32} & #1_{33} & \cdots & #1_{3n} \\
%	\vdots & \vdots & \vdots & \ddots & \vdots \\
%	#1_{m1} & #1_{m2} & #1_{m3} & \cdots & #1_{mn}
%	\end{bmatrix}		
%	}
	
\NewDocumentCommand{\matMN}{mg}{%
	\IfNoValueTF{#2}
		{%
		  \begin{bmatrix}
		   #1_{11} & #1_{12} & #1_{13} & \cdots & #1_{1n} \\
	       #1_{21} & #1_{22} & #1_{23} & \cdots & #1_{2n} \\
		   #1_{31} & #1_{32} & #1_{33} & \cdots & #1_{3n} \\
		   \vdots & \vdots & \vdots & \ddots & \vdots \\
		   #1_{m1} & #1_{m2} & #1_{m3} & \cdots & #1_{mn}
		   \end{bmatrix}	%
		}
		{%
		   \begin{bmatrix}
		   #1_{11} + #2_{11} & #1_{12} + #2_{12} & \cdots & #1_{1n} + #2_{2n} \\
		   #1_{21} + #2_{21} & #1_{22} + #2_{22} & \cdots & #1_{2n} + #2_{2n} \\
		   \vdots & \vdots & \vdots & \ddots & \vdots \\
		   #1_{m1} + #2_{m1} & #1_{m2} + #2_{m2} & \cdots & #1_{mn} + #2_{mn}
		   \end{bmatrix}
		}%
}

%%%%
\begin{document}
\thispagestyle{empty}

\begin{center}
\vspace*{2.5cm}

{\Huge \textsc{Math 307}}

\vspace*{2cm}

{\LARGE \textsc{Chapter 5}} 

\vspace*{0.75cm}

\noindent\textsc{Section 5.3: Matrices For Linear Transformations}

\vspace*{0.75cm}

\tableofcontents

\vfill

\noindent \textsc{Created by: Pierre-Olivier Paris{\'e}} \\
\textsc{Summer 2022}
\end{center}

\newpage

\section{Linear Transformation as A Matrix}

	\begin{example}\label{Ex:BasisMatrixOperator}
	Let $T : \mathbb{R}^3 \rightarrow \mathbb{R}^3$ be the linear transformation
		\begin{align*}
		T \left( \begin{bmatrix}
		x \\ y \\ z
		\end{bmatrix} \right) = \begin{bmatrix}
		5x + z \\ 3x + 2y - 3z \\ 5x
		\end{bmatrix} .
		\end{align*}
	Give a matrix representing the linear transformation $T$.
	\end{example}
	
	\newpage
	
	\underline{General Process}:

	Suppose $T : V \rightarrow W$ is a linear transformation. 
	
	\begin{itemize}
	\item Let $v_1$, $v_2$, $\ldots$, $v_n$ form a basis $\alpha$ for $V$.
	\item Let $w_1$, $w_2$, $\ldots$, $w_m$ form a basis $\beta$ for $W$.
	\end{itemize}
	
	Since $T(v_1)$, $T(v_2)$, $\ldots$, $T(v_n)$ belongs to $W$ and $\beta$ is a basis for $W$, we have
		\begin{align*}
		T(v_1) &= a_{11} w_1 + a_{21} w_2 + \cdots + a_{m1} w_m \\
		T(v_2) &= a_{12} w_1 + a_{22} w_2 + \cdots + a_{m2} w_m \\
		& \phantom{2222222222} \vdots \\
		T(v_n) &= a_{1n} w_1 + a_{2n} w_2 + \cdots + a_{mn} w_m .
		\end{align*}
	
	We call the \textbf{matrix of $\mathbf{T}$ with respect to the bases $\mathbf{\alpha}$ and $\mathbf{\beta}$} the matrix $[T]_{\alpha}^\beta$ formed from the previous coefficients $a_{11} , a_{22} , \ldots , a_{mn}$:
		\begin{align*}
		[T]_{\alpha}^\beta =
		\begin{bmatrix}
		a_{11} & a_{12} & \cdots & a_{1n} \\
		a_{21} & a_{22} & \cdots & a_{2n} \\
		\vdots & \vdots & \ddots & \vdots \\
		a_{m1} & a_{m2} & \cdots & a_{mn}
		\end{bmatrix} .
		\end{align*}
	
	\underline{Remarks}:
	\begin{itemize}
	\item With the notation introduced in Chapter 2 on basis, we have
		\begin{align*}
		[T]_{\alpha}^\beta = \begin{bmatrix}
		[T(v_1)]_{\beta} & [T(v_1)]_\beta & \cdots [T(v_n)]_\beta 
		\end{bmatrix} .
		\end{align*}
	\item When $T : V \rightarrow V$ is a linear transformation of $V$ into itself and $\alpha$ is used for both the domain and the codomain, then we simply say \textbf{the matrix of $\mathbf{T}$ with respect to $\mathbf{\alpha}$} and we denote it by $[T]_\alpha^\alpha$.
	\end{itemize}
	
	\newpage
	
	\begin{example}\label{Ex:MatrixOfTransfoOtherBasis}
	Let $T$ be the linear transformation in Example \ref{Ex:BasisMatrixOperator}. Let $\beta$ be the basis given by
		\begin{align*}
		\left\lbrace \begin{bmatrix}
		1 \\ 1 \\ 2
\end{bmatrix} , \begin{bmatrix}
1 \\ -1 \\ 1
\end{bmatrix} , \begin{bmatrix}
1 \\ 1 \\ 1
\end{bmatrix} \right\rbrace .
		\end{align*}
		Find
		\begin{enumerate}
		\item the matrix of $T$ with respect to the standard basis $\alpha$ of $\mathbb{R}^3$.
		\item the matrix of $T$ with respect to the basis $\beta$.
		\item $[T]_\alpha^\beta$.
		\end{enumerate}
	\end{example}
	
	\newpage
	
	\phantom{2}
	
	\vfill
	
	\section{Matrix of the Composition}
	Let $T : V \rightarrow W$ and $S : W \rightarrow U$ be linear transformations. Suppose that
		\begin{itemize}
		\item $\alpha$ is a basis for $V$;
		\item $\beta$ is a basis for $W$;
		\item $\gamma$ is a basis for $U$.
		\end{itemize}
	Then we have
		\begin{align*}
		[ST]_\alpha^\gamma = [S]_{\beta}^\gamma [T]_{\alpha}^\beta .
		\end{align*}
	
	\newpage
	
	\section{Matrix and Evaluation of Transformations}
	
	Given a transformation $T : V \rightarrow W$, a basis $\alpha$ for $V$ and a basis $\beta$ for $W$, we then have
		\begin{align*}
		[T(v)]_{\beta} = [T]_{\alpha}^\beta [v]_\alpha .
		\end{align*}
		
	\underline{Remark}: The last equality means that the vector $T(v)$ is obtained by multiplying the matrix of $T$ with respect to $\alpha$ and $\beta$ by the vector of the coordinates of $v$ in the basis $\alpha$.
	
	\vspace*{12pt}
	
	\begin{example}
	Let $T$, $\alpha$ and $\beta$ be as in Example \ref{Ex:MatrixOfTransfoOtherBasis}. 
		\begin{enumerate}
		\item Find the coordinate vector of $v = \begin{bmatrix} 1 & 2 & 3 \end{bmatrix}^\top$ with respect to the basis $\alpha$.
		\item Find coordinate vector of $T(v)$ with respect to the basis $\beta$.
		\item Use the result in part (b) to find $T(v)$ in the standard basis.
		\end{enumerate}
	\end{example}
	
	\newpage
	
	\section{Change of Basis}
	
	\subsection{Matrix of a Change of Basis}
	
	\begin{example}\label{Ex:ChangeBasis}
	Let $\alpha$ be the standard basis for $\mathbb{R}^3$ and let $\beta$ be the basis in Example \ref{Ex:MatrixOfTransfoOtherBasis}. Find a matrix that will send each vector in the basis $\alpha$ to the vectors in the basis $\beta$. 
	\end{example}
	
	\newpage
	
	\underline{General Procedure}:
	
	Let $\alpha$ and $\beta$ be two bases of $V$:
		\begin{itemize}
		\item $\alpha$ be a basis with vectors $v_1$, $v_2$, $\ldots$, $v_n$.
		\item $\beta$ be a basis with vectors $w_1$, $w_2$, $\ldots$, $w_n$.
		\end{itemize}
	Write
		\begin{align*}
		w_1 &= p_{11} v_1 + p_{21} v_2 + \cdots + p_{n1} v_n \\
		w_2 &= p_{12} v_1 + p_{22} v_2 + \cdots + p_{n2} v_n \\
		& \phantom{22222222222} \vdots \\
		w_n &= p_{1n} v_1 + p_{2n} v_2 + \cdots + p_{nn} v_n .
		\end{align*}
	Then the matrix
		\begin{align*}
		P = \begin{bmatrix}
		p_{11} & p_{12} & \cdots & p_{1n} \\
		p_{21} & p_{22} & \cdots & p_{2n} \\
		\vdots & \vdots & \ddots & \vdots \\
		p_{n1} & p_{n2} & \cdots & p_{nn}
		\end{bmatrix}
		\end{align*}
	is called the \textbf{change of basis matrix from $\mathbf{\alpha}$ to $\mathbf{\beta}$}.
	
	\underline{Fact}: 
	\begin{itemize}
	\item If we define $I(v) = v$ to be the identity transformation, then in fact $P = [I]_{\beta}^\alpha$. So, $[v]_{\alpha} = P [v]_\beta$.
	\vfill
	\item If $P$ is the change of basis matrix from a basis $\alpha$ to a basis $\beta$ of a vector space, then the change of basis from $\beta$ to $\alpha$ is $P^{-1}$. So $P^{-1} = [I]_\alpha^\beta$ and $[v]_\beta = P^{-1} [v]_\alpha$.
	\end{itemize}
	
	\newpage	
	
	\subsection{Consequence on the Matrix of a Linear Transformation}
	
	\begin{example}
	Let $\alpha$ be the standard basis and let $\beta$ be the basis in Example \ref{Ex:MatrixOfTransfoOtherBasis}. Suppose that a linear transformation $T$ has the following matrix with respect to $\alpha$:
		\begin{align*}
		[T]_{\alpha}^\alpha = \begin{bmatrix}
		1 & 3 & -1 \\ 2 & -1 & 2 \\ 1 & 3 & -1
		\end{bmatrix} .
		\end{align*}
	Find $[T(v)]_{\beta}$ where $[v]_{\beta} = \begin{bmatrix} 1 & 2 & 3 \end{bmatrix}^\top$. 
	\end{example}
	
	\newpage
	
	\phantom{222}
	
	\vfill
	
	\underline{Facts}:
	
	\begin{itemize}
	\item If $T : V \rightarrow V$ is a linear transformation, $\alpha$ and $\beta$ are bases for $V$, and $P$ is the change of basis matrix from $\alpha$ to $\beta$, then
		\begin{align*}
		[T]_\beta^\beta = P^{-1} [T]_\alpha^\alpha P .
		\end{align*}
	\item If $T : \mathbb{R}^n \rightarrow \mathbb{R}^m$ is a linear transformation and $A$ is the matrix of $T$ with respect to the standard basis of $\mathbb{R}^n$ and $\mathbb{R}^m$, then
		\begin{align*}
		T (X) = A X .
		\end{align*}
	\end{itemize}
	
\end{document}


	
	\underline{\textbf{Proof}}:
	
	\newpage
	
	\subsection{Change in the Matrix of a Transformation}
	Given a linear transformation $T : V \rightarrow W$, two bases $\alpha$ and $\alpha'$ of $V$ and two bases $\beta$ and $\beta'$ of $W$, if $P$ is the change of basis from $\alpha$ to $\alpha'$ and $Q$ is the change of basis from $\beta$ to $\beta'$, then
		\begin{align*}
		[T]_{\alpha'}^{\beta'} = Q^{-1} [T]_{\alpha}^\beta P .
		\end{align*}
