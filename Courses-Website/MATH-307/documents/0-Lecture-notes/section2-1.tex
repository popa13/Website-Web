\documentclass[12pt,a4paper]{article}
\usepackage[utf8]{inputenc}
\usepackage[english]{babel}

\usepackage{amsmath}
\usepackage{amsfonts}
\usepackage{amssymb}

\usepackage{graphicx}
\usepackage{lmodern}
\usepackage{tikz}
\usepackage{titlesec}
\usepackage{environ}
\usepackage{xcolor}
\usepackage{fancyhdr}
\usepackage[colorlinks = true, linkcolor = black]{hyperref}
\usepackage{xparse}
\usepackage{enumerate}

\usepackage[left=2cm,right=2cm,top=2cm,bottom=2cm]{geometry}
\usepackage{multicol}
\usepackage[indent=0pt]{parskip}

\newcommand{\spaceP}{\vspace*{0.5cm}}

%% Redefining sections
\newcommand{\sectionformat}[1]{%
    \begin{tikzpicture}[baseline=(title.base)]
        \node[rectangle, draw] (title) {#1};
    \end{tikzpicture}
    
    \noindent\hrulefill
}

% default values copied from titlesec documentation page 23
% parameters of \titleformat command are explained on page 4
\titleformat%
    {\section}% <command> is the sectioning command to be redefined, i. e., \part, \chapter, \section, \subsection, \subsubsection, \paragraph or \subparagraph.
    {\normalfont\large\scshape}% <format>
    {}% <label> the number
    {0em}% <sep> length. horizontal separation between label and title body
    {\centering\sectionformat}% code preceding the title body  (title body is taken as argument)

%% Set counters for sections to none
\setcounter{secnumdepth}{0}

%% Set the footer/headers
\pagestyle{fancy}
\fancyhf{}
\renewcommand{\headrulewidth}{0pt}
\renewcommand{\footrulewidth}{2pt}
\lfoot{P.-O. Paris{\'e}}
\cfoot{MATH 307}
\rfoot{Page \thepage}

%% Defining example environment
\newcounter{example}[section]
\NewEnviron{example}%
	{%
	\noindent\refstepcounter{example}\fcolorbox{gray!40}{gray!40}{\textsc{\textcolor{red}{Example~\theexample.}}}%
	%\fcolorbox{black}{white}%
		{  %\parbox{0.95\textwidth}%
			{
			\BODY
			}%
		}%
	}

% Theorem environment
\NewEnviron{theorem}%
	{%
	\noindent\refstepcounter{example}\fcolorbox{gray!40}{gray!40}{\textsc{\textcolor{blue}{Theorem~\theexample.}}}%
	%\fcolorbox{black}{white}%
		{  %\parbox{0.95\textwidth}%
			{
			\BODY
			}%
		}%
	}

%% Commands for matrix of dimensions m x n
%\newcommand{\matMN}[1]{%
%	\begin{bmatrix}
%	#1_{11} & #1_{12} & #1_{13} & \cdots & #1_{1n} \\
%	#1_{21} & #1_{22} & #1_{23} & \cdots & #1_{2n} \\
%	#1_{31} & #1_{32} & #1_{33} & \cdots & #1_{3n} \\
%	\vdots & \vdots & \vdots & \ddots & \vdots \\
%	#1_{m1} & #1_{m2} & #1_{m3} & \cdots & #1_{mn}
%	\end{bmatrix}		
%	}
	
\NewDocumentCommand{\matMN}{mg}{%
	\IfNoValueTF{#2}
		{%
		  \begin{bmatrix}
		   #1_{11} & #1_{12} & #1_{13} & \cdots & #1_{1n} \\
	       #1_{21} & #1_{22} & #1_{23} & \cdots & #1_{2n} \\
		   #1_{31} & #1_{32} & #1_{33} & \cdots & #1_{3n} \\
		   \vdots & \vdots & \vdots & \ddots & \vdots \\
		   #1_{m1} & #1_{m2} & #1_{m3} & \cdots & #1_{mn}
		   \end{bmatrix}	%
		}
		{%
		   \begin{bmatrix}
		   #1_{11} + #2_{11} & #1_{12} + #2_{12} & \cdots & #1_{1n} + #2_{2n} \\
		   #1_{21} + #2_{21} & #1_{22} + #2_{22} & \cdots & #1_{2n} + #2_{2n} \\
		   \vdots & \vdots & \vdots & \ddots & \vdots \\
		   #1_{m1} + #2_{m1} & #1_{m2} + #2_{m2} & \cdots & #1_{mn} + #2_{mn}
		   \end{bmatrix}
		}%
}

%%%%
\begin{document}
\thispagestyle{empty}

\begin{center}
\vspace*{2.5cm}

{\Huge \textsc{Math 307}}

\vspace*{2cm}

{\LARGE \textsc{Chapter 2}} 

\vspace*{0.75cm}

\noindent\textsc{Section 2.1: Vector Spaces}

\vspace*{0.75cm}

\tableofcontents

\vfill

\noindent \textsc{Created by: Pierre-Olivier Paris{\'e}} \\
\textsc{Summer 2022}
\end{center}

\newpage

\section{Examples}

	\subsection{Solutions to System of Linear Equations}
	
	\begin{example}
	Describe the set of solutions of the following system of linear equations.
		\begin{align*}
		2x + 3y - z & = 3 \\
		-x - y + 3z & = 0 \\
		x + 2y + 2z & = 3 \\
		y + 5z & = 3 .
		\end{align*}
	\end{example}
	
\newpage

\section{What is a Vector Space?}

	\subsection{Precise Definition}
	A nonempty set $V$ is a \textbf{vector space} if there are operations of addition (denoted by $+$ ) and scalar multiplication (denoted by $\cdot$ ) on $V$ such that the following eight properties are satisfied:
		\begin{enumerate}
		\item $u + v = v + u$ for any $u$ and $v$ in $V$;
		\item $u + (v + w) = (u + v) + w$ for any $u$, $v$, $w$ in $V$;
		\item There is an element denoted $0$ in $V$ so that $v + 0 = v$ for any $v$ in $V$.
		\item For each $v$ in $V$ there is an element denoted $-v$ so that $v + (-v) = 0$.
		\item $c \cdot (u + v) = c\cdot u + c \cdot v$ for all real number $c$ and for all $u$ and $v$ in $V$;
		\item $(c + d) \cdot v = c\cdot v + d\cdot v$ for all real numbers $c$ and $d$ and for all $v$ in $V$;
		\item $c \cdot (d\cdot v) = (cd) \cdot v$ for all real numbers $c$ and $d$ and for all $v$ in $V$;
		\item $1\cdot v = v$ for all $v$ in $V$.
		\end{enumerate}
		
	\noindent\underline{Remarks}:
		\begin{itemize}
		\item The eight above properties are called \textit{axioms}, \textit{postulates}, or \textit{laws} of a vector spaces.
		\item Don't confuse the abstract vectors from the more concrete column-vectors or row-vectors.
		\item The elements of the set $V$ are called \textit{vectors}.
		\item The real numbers are called \textit{scalars}.
		\end{itemize}
		
	\subsection{Column Vectors as a Vector space}
	\begin{example}
	The set of all $3 \times 1$ column vectors, denoted by $\mathbb{R}^3$, is a vector space if we define
		\begin{align*}
		\begin{bmatrix}
		x_1 \\ x_2 \\ x_3
		\end{bmatrix}
		+
		\begin{bmatrix}
		y_1 \\ y_2 \\ y_3
		\end{bmatrix}
		:=
		\begin{bmatrix}
		x_1 + y_1 \\
		x_2 + y_2 \\
		x_3 + y_3 
		\end{bmatrix}
		\quad \text{and} \quad
		c \cdot \begin{bmatrix}
		x_1 \\ x_2 \\ x_3
		\end{bmatrix}
		:= \begin{bmatrix}
		c x_1 \\ c x_2 \\ c x_3
		\end{bmatrix} .
		\end{align*}
	\end{example}
	
	\newpage
	
	\phantom{2}
	
	\vfill
	
	\begin{example}
	More generally, the set of all $n \times 1$ column vectors, denoted by $\mathbb{R}^n$ is a vector space if we define
	\begin{align*}
		\begin{bmatrix}
		x_1 \\ x_2 \\ \vdots \\ x_n
		\end{bmatrix}
		+
		\begin{bmatrix}
		y_1 \\ y_2 \\ \vdots \\ y_n
		\end{bmatrix}
		:=
		\begin{bmatrix}
		x_1 + y_1 \\
		x_2 + y_2 \\
		\vdots \\
		x_n + y_n
		\end{bmatrix}
		\quad \text{and} \quad
		c \cdot \begin{bmatrix}
		x_1 \\ x_2 \\ \vdots \\ x_n
		\end{bmatrix}
		:= \begin{bmatrix}
		c x_1 \\ c x_2 \\ \vdots \\ c x_n
		\end{bmatrix} .
		\end{align*}
	\end{example}
	
	\vspace*{12pt}
	
	\noindent\underline{Remark}: 
	\begin{itemize}
	\item The set of $1 \times n$ row vectors is also a vector space with the addition and scalar multiplication defined component-wise in a similar way.
	\end{itemize}
	
	\newpage
	
	\subsection{Matrices as a Vector Space}
	\begin{example}
	The set of $m \times n$ matrices $M_{m \times n} (\mathbb{R})$ is a vector space if we define the addition of two matrices and the scalar multiplication of a real number with a matrix by the matrix addition and matrix scalar multiplication defined in the previous chapter (see section 1.2).
	\end{example}
	
	\vspace*{6cm}
	
	\subsection{Functions as a Vector Space}
	\begin{example}
	Let $F(a, b)$ denote the set of all real-valued functions defined on $(a, b)$. Some examples are $f(x) = x^2$, $f(x) = \sin x$, $f(x) = |x|$, etc.
	
	We define the addition of two functions $f$ and $g$ to be the new function $(f + g)$ defined on $(a, b)$ by
		\begin{align*}
		(f + g) (x) := f(x) + g(x) .
		\end{align*}
	We define the scalar multiplication of a function $f$ with a real number $c$ to be the new function $(cf)$ defined on $(a, b)$ by
		\begin{align*}
		(cf) (x) := c f(x) .
		\end{align*}
	Show that $F(a, b)$ is a vector space.
	\end{example}
	
	\newpage
	
	\phantom{2}
	
	\newpage
	
	\subsection{A nonexample}
	\begin{example}
	Let $V$ be the set of $1 \times 2$ row vectors. We define an addition and a scalar multiplication by
		\begin{align*}
		\begin{bmatrix}
		x_1 & x_2
		\end{bmatrix}
		+ 
		\begin{bmatrix}
		y_1 & y_2
		\end{bmatrix}
		 := \begin{bmatrix}
		 x_1 + y_1 + 1 & x_2 + y_2
		 \end{bmatrix}
		 \quad \text{and} \quad
		 c \begin{bmatrix}
		 x_1 & x_2
		 \end{bmatrix}
		 := \begin{bmatrix}
		 cx_1 & c x_2 
		 \end{bmatrix} .
		\end{align*}
	Is $V$ equipped with these operations a vector space?
	\end{example}
	
\newpage

\phantom{2}

\newpage

\section{Simple Properties of Vector Spaces}

	\subsection{Uniqueness}
	Suppose that $V$ is a vector space.
		\begin{itemize}
		\item There is only one zero vector in $V$.
		\item If $v$ is a vector in $V$, there is only one negative (denoted by $-v$) of $v$.
		\end{itemize}
	
	\vspace*{18pt}
	
	\subsection{Multiplying by Zero}
	Let $V$ be a vector space.
		\begin{itemize}
		\item For any vector $v$ in $V$, we have $0 \cdot v = 0$.
		\item For any real number $c$, we have $c \cdot 0 = 0$.
		\end{itemize}
	
	\vspace*{18pt}
	
	\subsection{Subtraction in Vector space}
	
	Let $V$ be a vector space. Then for any vector $v$ in $V$, we have
		\begin{align*}
		(-1) \cdot v = -v .
		\end{align*}
	
	\vspace*{24pt}
	
	\underline{Remarks}:
		\begin{itemize}
		\item We usually write $c v$ instead of $c \cdot v$ for the scalar multiplication. It simplifies the notation.
		\item Subtracting two vectors is done in the following way: 
			\begin{align*}
			u - v := u + (-v) .
			\end{align*}
		\end{itemize}
	
\end{document}