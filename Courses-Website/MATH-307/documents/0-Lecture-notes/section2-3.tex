\documentclass[12pt,a4paper]{article}
\usepackage[utf8]{inputenc}
\usepackage[english]{babel}

\usepackage{amsmath}
\usepackage{amsfonts}
\usepackage{amssymb}

\usepackage{graphicx}
\usepackage{lmodern}
\usepackage{tikz}
\usepackage{titlesec}
\usepackage{environ}
\usepackage{xcolor}
\usepackage{fancyhdr}
\usepackage[colorlinks = true, linkcolor = black]{hyperref}
\usepackage{xparse}
\usepackage{enumerate}

\usepackage[left=2cm,right=2cm,top=2cm,bottom=2cm]{geometry}
\usepackage{multicol}
\usepackage[indent=0pt]{parskip}

\newcommand{\spaceP}{\vspace*{0.5cm}}
\newcommand{\Span}{\mathrm{Span}\,}

%% Redefining sections
\newcommand{\sectionformat}[1]{%
    \begin{tikzpicture}[baseline=(title.base)]
        \node[rectangle, draw] (title) {#1};
    \end{tikzpicture}
    
    \noindent\hrulefill
}

% default values copied from titlesec documentation page 23
% parameters of \titleformat command are explained on page 4
\titleformat%
    {\section}% <command> is the sectioning command to be redefined, i. e., \part, \chapter, \section, \subsection, \subsubsection, \paragraph or \subparagraph.
    {\normalfont\large\scshape}% <format>
    {}% <label> the number
    {0em}% <sep> length. horizontal separation between label and title body
    {\centering\sectionformat}% code preceding the title body  (title body is taken as argument)

%% Set counters for sections to none
\setcounter{secnumdepth}{0}

%% Set the footer/headers
\pagestyle{fancy}
\fancyhf{}
\renewcommand{\headrulewidth}{0pt}
\renewcommand{\footrulewidth}{2pt}
\lfoot{P.-O. Paris{\'e}}
\cfoot{MATH 307}
\rfoot{Page \thepage}

%% Defining example environment
\newcounter{example}[section]
\NewEnviron{example}%
	{%
	\noindent\refstepcounter{example}\fcolorbox{gray!40}{gray!40}{\textsc{\textcolor{red}{Example~\theexample.}}}%
	%\fcolorbox{black}{white}%
		{  %\parbox{0.95\textwidth}%
			{
			\BODY
			}%
		}%
	}

% Theorem environment
\NewEnviron{theorem}%
	{%
	\noindent\refstepcounter{example}\fcolorbox{gray!40}{gray!40}{\textsc{\textcolor{blue}{Theorem~\theexample.}}}%
	%\fcolorbox{black}{white}%
		{  %\parbox{0.95\textwidth}%
			{
			\BODY
			}%
		}%
	}

%% Commands for matrix of dimensions m x n
%\newcommand{\matMN}[1]{%
%	\begin{bmatrix}
%	#1_{11} & #1_{12} & #1_{13} & \cdots & #1_{1n} \\
%	#1_{21} & #1_{22} & #1_{23} & \cdots & #1_{2n} \\
%	#1_{31} & #1_{32} & #1_{33} & \cdots & #1_{3n} \\
%	\vdots & \vdots & \vdots & \ddots & \vdots \\
%	#1_{m1} & #1_{m2} & #1_{m3} & \cdots & #1_{mn}
%	\end{bmatrix}		
%	}
	
\NewDocumentCommand{\matMN}{mg}{%
	\IfNoValueTF{#2}
		{%
		  \begin{bmatrix}
		   #1_{11} & #1_{12} & #1_{13} & \cdots & #1_{1n} \\
	       #1_{21} & #1_{22} & #1_{23} & \cdots & #1_{2n} \\
		   #1_{31} & #1_{32} & #1_{33} & \cdots & #1_{3n} \\
		   \vdots & \vdots & \vdots & \ddots & \vdots \\
		   #1_{m1} & #1_{m2} & #1_{m3} & \cdots & #1_{mn}
		   \end{bmatrix}	%
		}
		{%
		   \begin{bmatrix}
		   #1_{11} + #2_{11} & #1_{12} + #2_{12} & \cdots & #1_{1n} + #2_{2n} \\
		   #1_{21} + #2_{21} & #1_{22} + #2_{22} & \cdots & #1_{2n} + #2_{2n} \\
		   \vdots & \vdots & \vdots & \ddots & \vdots \\
		   #1_{m1} + #2_{m1} & #1_{m2} + #2_{m2} & \cdots & #1_{mn} + #2_{mn}
		   \end{bmatrix}
		}%
}

%%%%
\begin{document}
\thispagestyle{empty}

\begin{center}
\vspace*{2.5cm}

{\Huge \textsc{Math 307}}

\vspace*{2cm}

{\LARGE \textsc{Chapter 2}} 

\vspace*{0.75cm}

\noindent\textsc{Section 2.3: Linear Independence and Bases}

\vspace*{0.75cm}

\tableofcontents

\vfill

\noindent \textsc{Created by: Pierre-Olivier Paris{\'e}} \\
\textsc{Summer 2022}
\end{center}

\newpage

\section{Linear Independence}

	\subsection{Definition}
	
	Suppose that $v_1$, $v_2$, $\ldots$, $v_n$ are vectors in a vector space $V$. 
	
	\begin{itemize}
	\item The vectors $v_1$, $v_2$, $\ldots$, $v_n$ are \textbf{linearly dependent} if there are scalars $c_1$, $c_2$, $\ldots$, $c_n$, not all zero, so that
		\begin{align*}
		c_1 v_1 + c_2 v_2 + \cdots + c_n v_n = 0 .
		\end{align*}
	\item If $v_1$, $v_2$, $\ldots$, $v_n$ are not linearly dependent, then the vectors are \textbf{linearly independent}.
	\end{itemize}
	
	\begin{example}
	Are the vectors
		\begin{align*}
		\begin{bmatrix}
		1 \\ 2 \\ 3
		\end{bmatrix},
		\begin{bmatrix}
		3 \\ 2 \\ 1
		\end{bmatrix} , 
		\begin{bmatrix}
		-1 \\ 2 \\ 5
		\end{bmatrix}
		\end{align*}
	linearly dependent or linearly independent?
	\end{example}
	
	\newpage
	
	\begin{example}
	Are $x^2 + 1$, $x^2 - x + 1$, $x + 2$ linearly dependent or linearly independent?
	\end{example}
	
	\vfill
	
	\underline{Remark}: To show that the vectors $v_1$, $v_2$, $\ldots$, $v_n$ are linearly independent, we can verify that the following implication is true:
		\begin{align*}
		\text{If } c_1 v_1 + c_2 v_2 + \cdots + c_n v_n = 0 \text{, then } c_1 = c_2 = \cdots = c_n = 0 .
		\end{align*}
	\newpage
	
	\subsection{Dependence and Linear Combination}
	
	A way to check if a bunch of vectors are linearly dependent is outlined in the following statement.
	
	\vspace*{18pt}
	
	\begin{theorem}
	Suppose $v_1$, $v_2$, $\ldots$, $v_n$ are vectors in a vector space $V$. Then $v_1$, $v_2$, $\ldots$, $v_n$ are linearly dependent if and only if one of $v_1$, $v_2$, $\ldots$, $v_n$ is a linear combination of the others.
	\end{theorem}
	
	\vspace*{18pt}
	
	\begin{example}
	Apply the last Theorem to show that the vectors
		\begin{align*}
		\begin{bmatrix}
		1 \\ -2 \\ 3
		\end{bmatrix}
		\text{ and }
		\begin{bmatrix}
		3 \\ -6 \\ 9
		\end{bmatrix}
		\text{ and }
		\begin{bmatrix}
		-1 \\ 3 \\ 0
		\end{bmatrix} .
		\end{align*}
	are linearly dependent.
	\end{example}
	
	\newpage
	
\section{What is a Basis?}

	\subsection{Definition}
	
	The vectors $v_1$, $v_2$, $\ldots$, $v_n$ of a vector space $V$ are a \textbf{basis} if the two following conditions are satisfied:
		\begin{itemize}
		\item $v_1$, $v_2$, $\ldots$, $v_n$ are linearly independent. [Indepence Condition or IC]
		\item $v_1$, $v_2$, $\ldots$, $v_n$ span $V$. [Spanning condition, or SC]
		\end{itemize}
		
	\begin{example}
	Show that the vectors
		\begin{align*}
		e_1 = \begin{bmatrix}
		1 \\ 0 \\ 0
		\end{bmatrix} , \quad
		e_2 = \begin{bmatrix}
		0 \\ 1 \\ 0
		\end{bmatrix} , \quad
		e_3 = \begin{bmatrix}
		0 \\ 0 \\ 1
		\end{bmatrix}
		\end{align*}
	forms a basis for $\mathbb{R}^3$.
	\end{example}
	
	\vfill
	
	\noindent\underline{Remark}: The basis in the last example is called the standard basis for $\mathbb{R}^3$. More generally, the vectors
		\begin{align*}
		e_1 = \begin{bmatrix}
		1 \\ 0 \\ 0 \\ \vdots \\ 0 \\ 0
		\end{bmatrix} , \quad
		e_2 = \begin{bmatrix}
		0 \\ 1 \\ 0 \\ \vdots \\ 0 \\ 0
		\end{bmatrix} , \quad
		e_3 = \begin{bmatrix}
		0 \\ 0 \\ 1 \\ \vdots \\ 0 \\ 0
		\end{bmatrix} , \quad 
		\ldots , \quad
		e_{n-1} = \begin{bmatrix}
		0 \\ 0 \\ 0 \\ \vdots \\ 1 \\ 0
		\end{bmatrix} , \quad
		e_{n} = \begin{bmatrix}
		0 \\ 0 \\ 0 \\ \vdots \\ 0 \\ 1
		\end{bmatrix}
		\end{align*}
	forms a basis for the vector space $\mathbb{R}^n$ of column vectors of dimensions $n \times 1$.
	
	\newpage
	
	\subsection{Basis for Matrices and Polynomials}
	
	\begin{example}
	The vectors
		\begin{align*}
		E_{11} = \begin{bmatrix}
		1 & 0 \\
		0 & 0 
		\end{bmatrix}
		, \quad
		E_{12} = \begin{bmatrix}
		0 & 1 \\
		0 & 0
		\end{bmatrix} 
		, \quad
		E_{21} = \begin{bmatrix}
		0 & 0 \\
		1 & 0
		\end{bmatrix}
		, \quad
		E_{22} = \begin{bmatrix}
		0 & 0 \\
		0 & 1
		\end{bmatrix}
		\end{align*}
	form a basis for the vector space of $2 \times 2$ matrices $M_{2 \times 2} (\mathbb{R})$.
	\end{example}
	
	\vspace*{20pt}
	
	\noindent\underline{Remark}: The basis in the last example is called the standard basis for the vector space $M_{2 \times 2} (\mathbb{R})$. More generally, the vectors $E_{ij}$ with a $1$ in the entry $ij$ and $0$ elsewhere forms a basis for the space of matrices $M_{m \times n } (\mathbb{R})$.
	
	\vspace*{25pt}
	
	\begin{example}
	The vectors
		\begin{align*}
		1, \, x , \, x^2
		\end{align*}
	form a basis for the set of polynomials $P_2$.
	\end{example}
	
	\vfill
	
	\noindent\underline{Remark}: The basis in the last example is also called the standard basis for the vector space $P_2$. More generally, for a nonnegative integer $n$, the vectors
		\begin{align*}
		x^n , \, x^{n-1} , \, \ldots , \, x , \, 1
		\end{align*}
	form a basis for the vector space $P_n$ of polynomials of degree less than or equal to $n$.
	
	\newpage
	
	\begin{example}\label{Ex:BasisOtherStandard}
	Do the vectors
		\begin{align*}
		\begin{bmatrix}
		1 \\ 0 \\ 1
		\end{bmatrix} , \quad
		\begin{bmatrix}
		1 \\ 1 \\ 1
		\end{bmatrix} , \quad
		\begin{bmatrix}
		-1 \\ 1 \\ 1
		\end{bmatrix}
		\end{align*}
	form a basis for the vector space of $3 \times 1$ column vectors?
	\end{example}
	
	%\newpage
	
	%\begin{example}
	%Do the vectors
		%\begin{align*}
		%\begin{bmatrix}
		%1 \\ 1
		%\end{bmatrix} , \quad
		%\begin{bmatrix}
		%-1 \\ 1
		%\end{bmatrix} , \quad
		%\begin{bmatrix}
		%2 \\ 3
		%\end{bmatrix}
		%\end{align*}
	%form a basis for $\mathbb{R}^2$?
	%\end{example}
	
	\newpage
	
	\subsection{Coordinates relative to a basis}
	
	In many applications, like robotics, it is really important to be able to represent the position of a moving part of a robot in terms of a new coordinates system. 
	
	\vspace*{14pt}
	
	\noindent Basis are an essential tools to do that. Given a basis $v_1$, $v_2$, $\ldots$, $v_n$ of a vector space $V$, each vector $v$ in $V$ can be expressed as a linear combination of the vectors in the basis:
		\begin{align}
		v = c_1 v_1 + c_2 v_2 + \cdots + c_n v_n . \label{Eq:LinearBasis}
		\end{align}
	Moreover, the scalars $c_1$, $c_2$, $\ldots$, $c_n$ in the Equation \eqref{Eq:LinearBasis} are unique. This means that there is only one list of scalars $c_1$, $c_2$, $\ldots$, $c_n$ that satisfies Equation \eqref{Eq:LinearBasis}.
	
	\vspace*{14pt}
	
	\begin{itemize}
	\item The list of scalars $c_1$, $c_2$, $\ldots$, $c_n$ are called the \textbf{coordinates of} $\mathbf{v}$ \textbf{relative to the basis} $\mathbf{v_1, \, v_2 , \, \ldots , \, v_n}$. 
	
	\item If $\alpha$ denotes the basis $v_1$, $v_2$, $\ldots$, $v_n$, then the column vector
		\begin{align*}
		[v]_{\alpha} = \begin{bmatrix}
		c_1 \\ c_2 \\ \vdots \\ c_n
		\end{bmatrix}
		\end{align*}
	is called the \textbf{coordinate vector of} $\mathbf{v}$ \textbf{relative to the basis} $\mathbf{\alpha}$.
	\end{itemize}
	
	\vspace*{18pt}
	
	\noindent{Remarks}:
		\begin{itemize}
		\item Coordinates relative to the standard basis:
			\vfill
		\item It is important to not confuse the column vectors representing the vector in a certain basis with the column vectors representing the vector in the standard basis.
		\end{itemize}
	
	\newpage
	
	\begin{example}
	Find the coordinate vector of
		\begin{align*}
		v = \begin{bmatrix}
		1 \\ 3 \\ 7
		\end{bmatrix}
		\end{align*}
	relative to the basis $\alpha$ for $\mathbb{R}^3$ presented in Example \ref{Ex:BasisOtherStandard}.
	\end{example}
	
	\newpage
	
	\begin{example}
	Find the coordinate in the standard basis of the vector $v$ in $\mathbb{R}^3$ if
		\begin{align*}
		[v]_{\alpha} = \begin{bmatrix}
		1 \\ 2 \\ -1
		\end{bmatrix}
		\end{align*}
	where $\alpha$ is the basis for $\mathbb{R}^3$ in Example \ref{Ex:BasisOtherStandard}.
	\end{example}
	
\end{document}