\documentclass[12pt,a4paper]{article}
\usepackage[utf8]{inputenc}
\usepackage[english]{babel}

\usepackage{amsmath}
\usepackage{amsfonts}
\usepackage{amssymb}

\usepackage{graphicx}
\usepackage{lmodern}
\usepackage{tikz}
\usepackage{titlesec}
\usepackage{environ}
\usepackage{xcolor}
\usepackage{fancyhdr}
\usepackage[colorlinks = true, linkcolor = black]{hyperref}
\usepackage{xparse}
\usepackage{enumerate}

\usepackage[left=2cm,right=2cm,top=2cm,bottom=2cm]{geometry}
\usepackage{multicol}

\newcommand{\spaceP}{\vspace*{0.5cm}}

%% Redefining sections
\newcommand{\sectionformat}[1]{%
    \begin{tikzpicture}[baseline=(title.base)]
        \node[rectangle, draw] (title) {#1};
    \end{tikzpicture}
    
    \noindent\hrulefill
}

% default values copied from titlesec documentation page 23
% parameters of \titleformat command are explained on page 4
\titleformat%
    {\section}% <command> is the sectioning command to be redefined, i. e., \part, \chapter, \section, \subsection, \subsubsection, \paragraph or \subparagraph.
    {\normalfont\large\scshape}% <format>
    {}% <label> the number
    {0em}% <sep> length. horizontal separation between label and title body
    {\centering\sectionformat}% code preceding the title body  (title body is taken as argument)

%% Set counters for sections to none
\setcounter{secnumdepth}{0}

%% Set the footer/headers
\pagestyle{fancy}
\fancyhf{}
\renewcommand{\headrulewidth}{0pt}
\renewcommand{\footrulewidth}{2pt}
\lfoot{P.-O. Paris{\'e}}
\cfoot{MATH 307}
\rfoot{Page \thepage}

%% Defining example environment
\newcounter{example}[section]
\NewEnviron{example}%
	{%
	\noindent\refstepcounter{example}\fcolorbox{gray!40}{gray!40}{\textsc{\textcolor{red}{Example~\theexample.}}}%
	%\fcolorbox{black}{white}%
		{  %\parbox{0.95\textwidth}%
			{
			\BODY
			}%
		}%
	}

% Theorem environment
\NewEnviron{theorem}%
	{%
	\noindent\refstepcounter{example}\fcolorbox{gray!40}{gray!40}{\textsc{\textcolor{blue}{Theorem~\theexample.}}}%
	%\fcolorbox{black}{white}%
		{  %\parbox{0.95\textwidth}%
			{
			\BODY
			}%
		}%
	}

%% Commands for matrix of dimensions m x n
%\newcommand{\matMN}[1]{%
%	\begin{bmatrix}
%	#1_{11} & #1_{12} & #1_{13} & \cdots & #1_{1n} \\
%	#1_{21} & #1_{22} & #1_{23} & \cdots & #1_{2n} \\
%	#1_{31} & #1_{32} & #1_{33} & \cdots & #1_{3n} \\
%	\vdots & \vdots & \vdots & \ddots & \vdots \\
%	#1_{m1} & #1_{m2} & #1_{m3} & \cdots & #1_{mn}
%	\end{bmatrix}		
%	}
	
\NewDocumentCommand{\matMN}{mg}{%
	\IfNoValueTF{#2}
		{%
		  \begin{bmatrix}
		   #1_{11} & #1_{12} & #1_{13} & \cdots & #1_{1n} \\
	       #1_{21} & #1_{22} & #1_{23} & \cdots & #1_{2n} \\
		   #1_{31} & #1_{32} & #1_{33} & \cdots & #1_{3n} \\
		   \vdots & \vdots & \vdots & \ddots & \vdots \\
		   #1_{m1} & #1_{m2} & #1_{m3} & \cdots & #1_{mn}
		   \end{bmatrix}	%
		}
		{%
		   \begin{bmatrix}
		   #1_{11} + #2_{11} & #1_{12} + #2_{12} & \cdots & #1_{1n} + #2_{2n} \\
		   #1_{21} + #2_{21} & #1_{22} + #2_{22} & \cdots & #1_{2n} + #2_{2n} \\
		   \vdots & \vdots & \vdots & \ddots & \vdots \\
		   #1_{m1} + #2_{m1} & #1_{m2} + #2_{m2} & \cdots & #1_{mn} + #2_{mn}
		   \end{bmatrix}
		}%
}

%%%%
\begin{document}
\thispagestyle{empty}

\begin{center}
\vspace*{2.5cm}

{\Huge \textsc{Math 307}}

\vspace*{2cm}

{\LARGE \textsc{Chapter 1}} 

\vspace*{0.75cm}

\noindent\textsc{Section 1.5: Determinants}

\vspace*{0.75cm}

\tableofcontents

\vfill

\noindent \textsc{Created by: Pierre-Olivier Paris{\'e}} \\
\textsc{Summer 2022}
\end{center}

\newpage

\section{Origin of the Determinant}

\begin{example}
Find the equation of the parabola $ax^2 + bx + 1$ passing through the points $(1, 1)$ and $(2, 4)$.
\end{example}

\vfill

\noindent\underline{Historical Notes}:
	\begin{itemize}
	\item Chinese scholars were the first to use determinants to solve systems of linear equations ($3^{\text{rd}}$ century BCE!).
	\item Cramer (1779) and Bezout (1779 also) used determinant to find a plane curve passing through a set of points, like we did in the previous example.
	\end{itemize}

\newpage

\section{Definition}

	\subsection{2 by 2 matrices}
	Given a $2 \times 2$ matrix 
		\begin{align*}
		A = \begin{bmatrix}
		a_{11} & a_{12} \\ a_{21} & a_{22}
		\end{bmatrix},
		\end{align*}
	the determinant of $A$, denoted by $\mathrm{det} \, (A)$ is
		\begin{align*}
		\mathrm{det}\, (A) = a_{11} a_{22} - a_{12} a_{21} .
		\end{align*}
		
	\noindent\underline{Remark}: Another notation for the determinant is
		\begin{align*}
		\det (A) = \begin{vmatrix}
		a_{11} & a_{12} \\ a_{21} & a_{22}
		\end{vmatrix} .
		\end{align*}
		
	\begin{example}
	Calculate the determinant of the following matrices:
		\begin{align*}
		A = \begin{bmatrix}
		1 & 2 \\ 3 & 4
		\end{bmatrix}
		\quad \text{and} \quad B = \begin{bmatrix}
		-1 & 2 \\ 1 & -2
		\end{bmatrix} .
		\end{align*}
	\end{example}
	
	\newpage
	
	\subsection{3 by 3 matrices}
	Let $A$ be a general $3 \times 3$ matrix
		\begin{align*}
		A = \begin{bmatrix}
		a_{11} & a_{12} & a_{13} \\ a_{21} & a_{22} & a_{32} \\
		a_{31} & a_{32} & a_{33}
		\end{bmatrix}
		\end{align*}
		
		\begin{itemize}
		\item \underline{Minor}: The minor of an entry $a_{ij}$ is the matrix $M_{ij}$ obtained from $A$ by removing row $i$ and column $j$.
		\item \underline{Cofactor}: The cofactor of an entry $a_{ij}$ is the matrix $C_{ij}$ given by
			\begin{align*}
			C_{ij} = (-1)^{i + j} \det (M_{ij}) .
			\end{align*}
		\end{itemize}
		
	\begin{example}
	Find the minor $M_{11}$, and the cofactor $C_{32}$ of the following matrices:
	\begin{align*}
		A = \begin{bmatrix}
		2 & 3 & -2 \\ -1 & 6 & 3 \\ 4 & -2 & 1
		\end{bmatrix}
		\quad \text{and} \quad
		B = \begin{bmatrix}
		1 & -2 & 2 \\ 2 & -3 & -1 \\ 1 & -1 & -1
		\end{bmatrix} .
		\end{align*}
	\end{example}
	
	\newpage
	
	\noindent The determinant of $A$ is given by
		\begin{align*}
		\det (A) &= a_{11} \det (M_{11}) - a_{12} \det (M_{12}) + a_{13} \det (M_{13})   \\
		&=a_{11}
		\begin{vmatrix}
		a_{22} & a_{23} \\ a_{32} & a_{33}
		\end{vmatrix}
		- a_{12}
		\begin{vmatrix}
		a_{21} & a_{23} \\ a_{31} & a_{33}
		\end{vmatrix}
		+ a_{13}
		\begin{vmatrix}
		a_{12} & a_{22} \\ a_{31} & a_{32}
		\end{vmatrix} \\
		& = a_{11} a_{22} a_{33} - a_{11} a_{23} a_{32} - a_{12} a_{21} a_{33} + a_{12} a_{23} a_{32} + a_{13} a_{12}a_{32} - a_{13} a_{22} a_{31} .
		\end{align*}

	\begin{example}\label{Example:3x3Determinants}
	Find the determinant of the following matrices:
		\begin{align*}
		A = \begin{bmatrix}
		2 & 3 & -2 \\ -1 & 6 & 3 \\ 4 & -2 & 1
		\end{bmatrix}
		\quad \text{and} \quad
		B = \begin{bmatrix}
		1 & -2 & 2 \\ 2 & -3 & -1 \\ 1 & -1 & -1
		\end{bmatrix} .
		\end{align*}
	\end{example}

	\newpage
	
	\subsection{For General Matrices}
	The determinant is defined recursively.
		\begin{enumerate}
		\item If $A$ is an $2 \times 2$ matrix, then $\det (A) = a_{11} a_{22} - a_{12}a_{21}$.
		\item If $A$ is an $n \times n$ matrix, then
			\begin{align*}
			\det (A) &= a_{11} C_{11} + a_{12} C_{12} + \cdots + a_{1n} C_{1n} \\
			&= a_{11} \det (M_{11}) - a_{12} \det (M_{12}) + \cdots + (-1)^{1 + n} a_{1n} \det (M_{1n}) .
			\end{align*}
		\end{enumerate}
		
		\begin{example}\label{Example:4x4Determinant}
		Compute the determinant of the following matrix:
			\begin{align*}
			A = \begin{bmatrix}
			7 & -3 & 0 & 4 \\
			0 & 1 & 0 & 3 \\
			2 & 1 & -2 & -5 \\
			0 & 4 & 0 & 6
			\end{bmatrix} .
			\end{align*}
		\end{example}
		
	\newpage
	
	\subsection{Determinant from any row or column}
	
	\noindent\underline{Lagrange's Expansion Formula}:
	If $A$ is an $n \times n$ matrix with $n \geq 2$, then
		\begin{itemize}
		\item $\det (A) = \sum_{j = 1}^n a_{ij} C_{ij}$ for any row indexed by $i$.
		\item $\det (A) = \sum_{i = 1}^n a_{ij} C_{ij}$ for any column indexed by $j$.
		\end{itemize}
	
	\begin{example}
	Compute again the determinant of the matrix $A$ in Example \ref{Example:3x3Determinants} by
		\begin{enumerate}
		\item expanding with respect to another row.
		\item expanding with respect to one of the column.
		\end{enumerate}
	\end{example}
	
	\vfill
	
	\noindent\underline{Advice}: It would be clever to choose the row or column containing the greatest number of zeros.
	
	\newpage
	
\section{Important Properties of Determinants}

	\subsection{When there too many zeros...}
		
	\begin{example}
	Find the determinant
		\begin{align*}
		A = \begin{vmatrix}
		1 & 1 & 1 & 0 & 1 & 1 & 1 \\
		1 & 0 & 0 & 0 & 1 & 0 & 0 \\
		0 & 1 & -1 & 0 & 1 & 2 & 0 \\
		-1 & 2 & 3 & 0 & 4 & 2 & -1 \\
		-2 & 2 & -1 & 0 & 3 & -3 & -1 \\
		1 & -1 & 1 & 0 & 3 & -2 & -1 \\
		-2 & -1 & 1 & 0 & -5 & 1 & -6
		\end{vmatrix}.
		\end{align*}
	\end{example}
	
	\vfill
	
	\noindent\underline{Fact}:
	If a matrix $A$ has a row or a column of zeros, then $\det (A) = 0$.
	
	\newpage
	
	\subsection{When the type matters!}
	
	\begin{example}
	Find the determinant
		\begin{align*}
		A = \begin{vmatrix}
		1 & 4 & 10 & 123 \\
		0 & 2 & 124 & \pi \\
		0 & 0 & 3 & \sqrt{2} \\
		0 & 0 & 0 & 4
		\end{vmatrix} .
		\end{align*}
	\end{example}
	
	\vfill
	
	\noindent\underline{Fact}:
	The determinant of a triangular (upper or lower) is the product of its diagonal entries.
	
	\newpage
	
	\subsection{When Operations matter!}
	
	When $E$ is an elementary matrix, 
	
		\begin{itemize}
		\item If $E$ switches row $i$ with row $j$, then $\det (E) = -1$.
		\item If $E$ is obtained from $I$ by multiplying a row by some scalar $c$, then $\det (E) = c$.
		\item If $E$ is obtained from $I$ by replacing a row of $I$ by itself plus a multiple of another row of $I$, then $\det (E) = 1$.
		\end{itemize}
	
	\noindent This implies the following general facts:
	Suppose that $A = [a_{ij}]$ is an $n \times n$ matrix with $n \geq 2$.
		\begin{itemize}
		\item If $B$ is a matrix obtained from $A$ by interchanging two rows of $A$, then $\det (B) = - \det (A)$.
		\item If $B$ is a matrix obtained from $A$ by multiplying a row of $A$ by a scalar $c$, then $\det (B) = c \det (A)$.
		\item If $B$ is a matrix obtained from $A$ by replacing a row of $A$ by itself plus a mlutiple of another row of $A$, then $\det (B) = \det (A)$. 
		\end{itemize}

\end{document}