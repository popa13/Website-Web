\documentclass[12pt,a4paper]{article}
\usepackage[utf8]{inputenc}
\usepackage[english]{babel}

\usepackage{amsmath}
\usepackage{amsfonts}
\usepackage{amssymb}

\usepackage{graphicx}
\usepackage{lmodern}
\usepackage{tikz}
\usepackage{titlesec}
\usepackage{environ}
\usepackage{xcolor}
\usepackage{fancyhdr}
\usepackage[colorlinks = true, linkcolor = black]{hyperref}
\usepackage{xparse}
\usepackage{enumerate}

\usepackage[left=2cm,right=2cm,top=2cm,bottom=2cm]{geometry}
\usepackage{multicol}
\usepackage[indent=0pt]{parskip}

\newcommand{\spaceP}{\vspace*{0.5cm}}
\newcommand{\Span}{\mathrm{Span}\,}

%% Redefining sections
\newcommand{\sectionformat}[1]{%
    \begin{tikzpicture}[baseline=(title.base)]
        \node[rectangle, draw] (title) {#1};
    \end{tikzpicture}
    
    \noindent\hrulefill
}

% default values copied from titlesec documentation page 23
% parameters of \titleformat command are explained on page 4
\titleformat%
    {\section}% <command> is the sectioning command to be redefined, i. e., \part, \chapter, \section, \subsection, \subsubsection, \paragraph or \subparagraph.
    {\normalfont\large\scshape}% <format>
    {}% <label> the number
    {0em}% <sep> length. horizontal separation between label and title body
    {\centering\sectionformat}% code preceding the title body  (title body is taken as argument)

%% Set counters for sections to none
\setcounter{secnumdepth}{0}

%% Set the footer/headers
\pagestyle{fancy}
\fancyhf{}
\renewcommand{\headrulewidth}{0pt}
\renewcommand{\footrulewidth}{2pt}
\lfoot{P.-O. Paris{\'e}}
\cfoot{MATH 307}
\rfoot{Page \thepage}

%% Defining example environment
\newcounter{example}[section]
\NewEnviron{example}%
	{%
	\noindent\refstepcounter{example}\fcolorbox{gray!40}{gray!40}{\textsc{\textcolor{red}{Example~\theexample.}}}%
	%\fcolorbox{black}{white}%
		{  %\parbox{0.95\textwidth}%
			{
			\BODY
			}%
		}%
	}

% Theorem environment
\NewEnviron{theorem}%
	{%
	\noindent\refstepcounter{example}\fcolorbox{gray!40}{gray!40}{\textsc{\textcolor{blue}{Theorem~\theexample.}}}%
	%\fcolorbox{black}{white}%
		{  %\parbox{0.95\textwidth}%
			{
			\BODY
			}%
		}%
	}

%% Commands for matrix of dimensions m x n
%\newcommand{\matMN}[1]{%
%	\begin{bmatrix}
%	#1_{11} & #1_{12} & #1_{13} & \cdots & #1_{1n} \\
%	#1_{21} & #1_{22} & #1_{23} & \cdots & #1_{2n} \\
%	#1_{31} & #1_{32} & #1_{33} & \cdots & #1_{3n} \\
%	\vdots & \vdots & \vdots & \ddots & \vdots \\
%	#1_{m1} & #1_{m2} & #1_{m3} & \cdots & #1_{mn}
%	\end{bmatrix}		
%	}
	
\NewDocumentCommand{\matMN}{mg}{%
	\IfNoValueTF{#2}
		{%
		  \begin{bmatrix}
		   #1_{11} & #1_{12} & #1_{13} & \cdots & #1_{1n} \\
	       #1_{21} & #1_{22} & #1_{23} & \cdots & #1_{2n} \\
		   #1_{31} & #1_{32} & #1_{33} & \cdots & #1_{3n} \\
		   \vdots & \vdots & \vdots & \ddots & \vdots \\
		   #1_{m1} & #1_{m2} & #1_{m3} & \cdots & #1_{mn}
		   \end{bmatrix}	%
		}
		{%
		   \begin{bmatrix}
		   #1_{11} + #2_{11} & #1_{12} + #2_{12} & \cdots & #1_{1n} + #2_{2n} \\
		   #1_{21} + #2_{21} & #1_{22} + #2_{22} & \cdots & #1_{2n} + #2_{2n} \\
		   \vdots & \vdots & \vdots & \ddots & \vdots \\
		   #1_{m1} + #2_{m1} & #1_{m2} + #2_{m2} & \cdots & #1_{mn} + #2_{mn}
		   \end{bmatrix}
		}%
}

%%%%
\begin{document}
\thispagestyle{empty}

\begin{center}
\vspace*{2.5cm}

{\Huge \textsc{Math 307}}

\vspace*{2cm}

{\LARGE \textsc{Chapter 2}} 

\vspace*{0.75cm}

\noindent\textsc{Section 2.4: Dimension; Nullspace, Row Space, and Column Space}

\vspace*{0.75cm}

\tableofcontents

\vfill

\noindent \textsc{Created by: Pierre-Olivier Paris{\'e}} \\
\textsc{Summer 2022}
\end{center}

\newpage

\section{What is the Dimension of a Vector Space?}

	\subsection{Definition}
	If a vector space $V$ has a basis $\alpha$ of $n$ vectors $v_1$, $v_2$, $\ldots$ , $v_n$, then the \textbf{dimension} of $V$ is the number of elements in the basis $\alpha$. 
	
	\vspace*{8pt}
	
	The definition comes from the following result telling us that all basis have the same number of elements.
	
	\vspace*{16pt}
	
	\begin{theorem}
	If $v_1$, $v_2$, $\ldots$, $v_n$ and $w_1$, $w_2$, $\ldots$, $w_n$ both form bases for a vector space $V$, then $n = m$.
	\end{theorem}
	
	\noindent\underline{Remark}:
		\begin{itemize}
		\item We use the symbol $\dim (V)$ to denote the dimension of a vector space $V$. 
		\item If, for a vector space $V$, $\dim (V)$ is a finite number, then $V$ is said to be \textbf{finite dimensional}.
		\item If, for a vector space $V$, $\dim (V)$ is infinite (so equal to $+\infty$), then $V$ is said to be \textbf{infinite dimensional}.
		\end{itemize}
		
	\begin{example}
	Which of the following vector space is finite dimensional and which one is infinite dimensional?
		\begin{enumerate}
		\begin{multicols}{2}
		\item $\mathbb{R}^3$.
		\item $\mathbb{R}^n$.
		\item $P_n$.
		\item $P$.
		\end{multicols}
		\end{enumerate}
	\end{example}
	
	\newpage
	
	\subsection{Basic Facts on Basis and Dimension}
	For a set of vectors to be a basis, it must have the same number of elements as the dimension of the vector space. 
	
	\vspace*{12pt}
	
	When there are not enough vectors but we know that there are linearly independant, we can add a set of new vectors so that the new set composed of the old vectors and the new vectors become a basis.
	
	\vspace*{16pt}
	
	\begin{example}
	Let $\alpha$ be the set of vectors
		\begin{align*}
		\alpha = \left\lbrace \begin{bmatrix}
		1 \\ 1 \\ 0
		\end{bmatrix}, \, \begin{bmatrix}
		0 \\ 1 \\ 1
		\end{bmatrix} \right\rbrace .
		\end{align*}
	Complete the set $\alpha$ with a certain number of vectors so that it becomes a basis of $\mathbb{R}^3$.
	\end{example}
	
	\vfill
	
	\noindent\underline{Remark}: Given a vector space $V$ of dimension $n$,
		\begin{itemize}
		\item if $v_1$, $v_2$, $\ldots$, $v_k$ are linearly independent vectors in $V$, then there exist vectors $v_{k+1}$, $\ldots$, $v_n$ so that $v_1$, $v_2$, $\ldots$, $v_k$, $v_{k+1}$, $\ldots$, $v_n$ form a basis for $V$.
		\end{itemize}
	
	\newpage
	
	When there are to much vectors, but we know that the set of vectors span $V$, we can extract a basis from it.
	
	\vspace*{16pt}
	
	\begin{example}
	Let $\alpha$ be the set
		\begin{align*}
		\alpha = \left\lbrace
			\begin{bmatrix}
			1 \\ 0 \\ 0
			\end{bmatrix} , \,
			\begin{bmatrix}
			0 \\ 1 \\ 1
			\end{bmatrix} , \,
			\begin{bmatrix}
			1 \\ 1 \\ 0
			\end{bmatrix} , \,
			\begin{bmatrix}
			1 \\ 0 \\ 1
			\end{bmatrix}
			\right\rbrace .
		\end{align*}
	Extract a basis from the set $\alpha$.
	\end{example}
	
	\vfill
	
	\noindent\underline{Remark}: Given a vector space $V$ of dimension $n$,
		\begin{itemize}
		\item if $v_1$, $v_2$, $\ldots$, $v_k$ span $V$, then there exists a subset of $v_1$, $v_2$, $\ldots$, $v_k$ that forms a basis of $V$.
		\end{itemize}
		
	\newpage
	
	\begin{theorem}
	Suppose that $V$ is a vector space of dimension $n$.
		\begin{itemize}
		\item If the vectors $v_1$, $v_2$, $\ldots$, $v_n$ are linearly independent, then $v_1$, $v_2$, $\ldots$, $v_n$ form a basis for $V$.
		\item If the vectors $v_1$, $v_2$, $\ldots$, $v_n$ span $V$, then $v_1$, $v_2$, $\ldots$, $v_n$ form a basis for $V$.
		\end{itemize}
	\end{theorem}
	
	\vspace*{16pt}
	
	\begin{example}
	Show that $x^2 - 1$, $x^2 + 1$, $x+1$ form a basis for $P_2$.
	\end{example}
	
\end{document}

\section{Subspaces Associated with a Matrix}

	\subsection{Null Space}
	The set of solutions to the homogeneous system
		\begin{align*}
		AX = O_{m \times 1}
		\end{align*}
	where $A$ is an $m \times n$ matrix for a subspace of $\mathbb{R}^n$. This subspace is called the \textbf{nullspace} or \textbf{kernel} of the matrix $A$ and is denoted by
		\begin{align*}
		NS (A) \quad \text{ or } \quad \ker (A) .
		\end{align*}
		
	\begin{example}\label{Ex:NullSpace}
	Find a basis for $NS (A)$ if
		\begin{align*}
		A = \begin{bmatrix}
		1 & 2 & -1 & 3 & 0 \\
		1 & 1 & 0 & 4 & 1 \\
		1 & 4 & -3 & 1 & -2
		\end{bmatrix} .
		\end{align*}
	\end{example}

	\newpage
	
	\subsection{Row Space}
	Given an $m \times n$ matrix $A$, the subspace spanned by the rows of $A$ is called the \textbf{row space} of $A$ and is denoted by
		\begin{align*}
		RS (A) .
		\end{align*}
		
	\vspace*{12pt}
	
	We will say that a matrix $A$ is row equivalent to a matrix $B$ if we can obtain $B$ from $A$ by a sequence of row operations on $A$.
	
	\vspace*{12pt}
	
	\begin{theorem}
	If $A$ is row equivalent to $B$, then $RS (A) = RS (B)$.
	\end{theorem}
	
	\vspace*{12pt}
	
	\begin{example}
	Use the last Theorem to find a basis for the row space of the matrix in Example \ref{Ex:NullSpace}.
	\end{example}
	
	\newpage
	
	\subsection{Relationship Between Null and Row Subspaces}
	Notice that
		\begin{itemize}
		\item the dimension of $NS (A)$ is the number of free variables in the solutions of $AX = O_{m \times 1}$.
		\item the dimension of $RS (A)$ is the number of nonzero rows of the RREF of $A$ which is the same as the number of nonfree variable in the solutions $AX = O_{m \times 1}$.
		\end{itemize}
	
	If you sum $\dim (NS(A))$ with $\dim (RS (A))$, we obtain the number of columns of $A$.
	
	\vspace*{12pt}
	
	\begin{theorem}
	If $A$ is an $m \times n$ matrix, then
		\begin{align*}
		\dim (RS (A)) + \dim (NS(A)) = n .
		\end{align*}
	\end{theorem}
	
	\subsection{Column Subspace}
	The subspace spanned by the column of an $m \times n$ matrix is called the \textbf{column space} and it is denoted by
		\begin{align*}
		CS (A) .
		\end{align*}
		
	\begin{example}
	Find a basis for the column space of the matrix
		\begin{align*}
		A = \begin{bmatrix}
		1 & 2 & -1 & 3 & 0 \\
		1 & 1 & 0 & 4 & 1 \\
		1 & 4 & -3 & 1 & -2
		\end{bmatrix} .
		\end{align*}
	\end{example}
	
	\newpage
	
	\phantom{c}
	
	\vfill
	
	In summary, to find $CS (A)$:
		\begin{itemize}
		\item Since we want to find the spanning set of the columns of $A$, we have to use column operations to reduce find a basis.
		\item Doing column operations on $A$ is the same thing as doing row operations on $A^\top$.
		\item So, we conclude that $CS (A) = RS (A^\top)$.
		\item We simply have to find the RREF of $A^\top$ and transpose the vectors spanning $RS (A^\top )$ to obtain a basis of $CS (A)$.
		\end{itemize}
	
	\underline{Fact}: We call $\dim (RS (A))$ which is the same as $\dim (CS(A))$ the \textbf{rank} of the matrix $A$.