\documentclass[12pt,a4paper]{article}
\usepackage[utf8]{inputenc}
\usepackage[english]{babel}

\usepackage{amsmath}
\usepackage{amsfonts}
\usepackage{amssymb}

\usepackage{graphicx}
\usepackage{lmodern}
\usepackage{tikz}
\usepackage{titlesec}
\usepackage{environ}
\usepackage{xcolor}
\usepackage{fancyhdr}
\usepackage[colorlinks = true, linkcolor = black]{hyperref}
\usepackage{xparse}
\usepackage{enumerate}

\usepackage[left=2cm,right=2cm,top=2cm,bottom=2cm]{geometry}
\usepackage{multicol}

\newcommand{\spaceP}{\vspace*{0.5cm}}

%% Redefining sections
\newcommand{\sectionformat}[1]{%
    \begin{tikzpicture}[baseline=(title.base)]
        \node[rectangle, draw] (title) {#1};
    \end{tikzpicture}
    
    \noindent\hrulefill
}

% default values copied from titlesec documentation page 23
% parameters of \titleformat command are explained on page 4
\titleformat%
    {\section}% <command> is the sectioning command to be redefined, i. e., \part, \chapter, \section, \subsection, \subsubsection, \paragraph or \subparagraph.
    {\normalfont\large\scshape}% <format>
    {}% <label> the number
    {0em}% <sep> length. horizontal separation between label and title body
    {\centering\sectionformat}% code preceding the title body  (title body is taken as argument)

%% Set counters for sections to none
\setcounter{secnumdepth}{0}

%% Set the footer/headers
\pagestyle{fancy}
\fancyhf{}
\renewcommand{\headrulewidth}{0pt}
\renewcommand{\footrulewidth}{2pt}
\lfoot{P.-O. Paris{\'e}}
\cfoot{MATH 307}
\rfoot{Page \thepage}

%% Defining example environment
\newcounter{example}[section]
\NewEnviron{example}%
	{%
	\noindent\refstepcounter{example}\fcolorbox{gray!40}{gray!40}{\textsc{\textcolor{red}{Example~\theexample.}}}%
	%\fcolorbox{black}{white}%
		{  %\parbox{0.95\textwidth}%
			{
			\BODY
			}%
		}%
	}

% Theorem environment
\NewEnviron{theorem}%
	{%
	\noindent\refstepcounter{example}\fcolorbox{gray!40}{gray!40}{\textsc{\textcolor{blue}{Theorem~\theexample.}}}%
	%\fcolorbox{black}{white}%
		{  %\parbox{0.95\textwidth}%
			{
			\BODY
			}%
		}%
	}

%% Commands for matrix of dimensions m x n
%\newcommand{\matMN}[1]{%
%	\begin{bmatrix}
%	#1_{11} & #1_{12} & #1_{13} & \cdots & #1_{1n} \\
%	#1_{21} & #1_{22} & #1_{23} & \cdots & #1_{2n} \\
%	#1_{31} & #1_{32} & #1_{33} & \cdots & #1_{3n} \\
%	\vdots & \vdots & \vdots & \ddots & \vdots \\
%	#1_{m1} & #1_{m2} & #1_{m3} & \cdots & #1_{mn}
%	\end{bmatrix}		
%	}
	
\NewDocumentCommand{\matMN}{mg}{%
	\IfNoValueTF{#2}
		{%
		  \begin{bmatrix}
		   #1_{11} & #1_{12} & #1_{13} & \cdots & #1_{1n} \\
	       #1_{21} & #1_{22} & #1_{23} & \cdots & #1_{2n} \\
		   #1_{31} & #1_{32} & #1_{33} & \cdots & #1_{3n} \\
		   \vdots & \vdots & \vdots & \ddots & \vdots \\
		   #1_{m1} & #1_{m2} & #1_{m3} & \cdots & #1_{mn}
		   \end{bmatrix}	%
		}
		{%
		   \begin{bmatrix}
		   #1_{11} + #2_{11} & #1_{12} + #2_{12} & \cdots & #1_{1n} + #2_{2n} \\
		   #1_{21} + #2_{21} & #1_{22} + #2_{22} & \cdots & #1_{2n} + #2_{2n} \\
		   \vdots & \vdots & \vdots & \ddots & \vdots \\
		   #1_{m1} + #2_{m1} & #1_{m2} + #2_{m2} & \cdots & #1_{mn} + #2_{mn}
		   \end{bmatrix}
		}%
}

%%%%
\begin{document}
\thispagestyle{empty}

\begin{center}
\vspace*{2.5cm}

{\Huge \textsc{Math 307}}

\vspace*{2cm}

{\LARGE \textsc{Chapter 1}} 

\vspace*{0.75cm}

\noindent\textsc{Section 1.4: Special Matrices and Additional Properties}

\vspace*{0.75cm}

\tableofcontents

\vfill

\noindent \textsc{Created by: Pierre-Olivier Paris{\'e}} \\
\textsc{Summer 2022}
\end{center}

\newpage

\section{Diagonal Matrices}
A diagonal matrix is a square matrix whose off diagonal entries are zero.

	\begin{itemize}
	\item \underline{Remark}: We denote a diagonal matrix by $\mathrm{diag}\,  (d_1 , d_2 , \ldots , d_n )$.
	\end{itemize}
	
	\begin{example}
	Give some examples of diagonal matrices.
	\end{example}
	
	\vfill
	
	\begin{example}
	Suppose $A = \mathrm{diag} (1, 2, -4, 3, 5 )$ and $B = \mathrm{diag} (-1 , 2, 0, 4, 3)$.
	\begin{multicols}{3}
		\begin{enumerate}[1)]
		\item Is $A$ invertible?
		\item Is $A + B$ invertible?
		\item Is $AB$ invertible?
		\end{enumerate}
	\end{multicols}
	\end{example}
	
	\vfill
	
	\newpage
	
	\phantom{2}
	
	\vfill
	
	\noindent\underline{General Facts}:
	Suppose $A$ and $B$ are diagonal matrices
		\begin{align*}
		A = \mathrm{diag}\, (a_1 , a_2 , \ldots , a_n ) \quad \text{ and } \quad B = \mathrm{diag}\, (b_1 , b_2 , \ldots , b_n ) .
		\end{align*}
	\begin{itemize}
	\item $A + B = \mathrm{diag} \, (a_1 + b_1 , a_2 + b_2 , \ldots , a_n + b_n )$.
	\item $AB = \mathrm{diag} \, (a_1 b_1 , a_2 b_2 , \ldots , a_n b_n )$.
	\item $A$ is invertible if and only if $a_i \neq 0$ for each $i$. In this case, we have 
		$$
		A^{-1} = \mathrm{diag}\, ( 1/a_1 , 1/a_2 , \ldots , 1/a_n ).
		$$
	\end{itemize}	
	
\newpage
	
\section{Triangular Matrices}

	\begin{itemize}
	\item \underline{Upper Triangular}: Square matrices whose entries below the diagonal are zero.
	\item \underline{Lower Triangular}: Square matrices whose entries above the diagonal are zero.
	\end{itemize}
	
	\begin{example}
	Give an example of an upper triangular matrix and an example of a lower triangular matrix.
	\end{example}
	

\newpage

\noindent\underline{General Facts}:
		\begin{itemize}
		\item If $A$ and $B$ are both upper triangular, then so is $A + B$; similarly if $A$ and $B$ are both lower triangular, then so is $A + B$.
		\item If $A$ and $B$ are both upper triangular, then so is $AB$; similarly if $A$ and $B$ are both lower triangular, then so is $AB$.
		\item $A$ is invertible if and only if each of the diagonal entries of $A$ is nonzero.
		\end{itemize}


	\begin{example}
	Let $A$ and $B$ be the two following $3 \times 3$ matrices:
		\begin{align*}
		A = \begin{bmatrix}
		1 & 2 & 3 \\
		0 & 4 & 5 \\
		0 & 0 & 7
		\end{bmatrix}
		\quad \text{ and } \quad
		B = \begin{bmatrix}
		-1 & 4 & 1 \\
		0 & -1 & 2 \\
		0 & 0 & 0 
		\end{bmatrix} .
		\end{align*}
	\begin{multicols}{3}
		\begin{enumerate}
		\item Is $A$ invertible?
		\item Is $B$ invertible?
		\item Is $AB$ upper or lower triangular matrix?
		\end{enumerate}
	\end{multicols}
	\end{example}
	
\newpage
	
\section{Symmetric Matrices}

\begin{itemize}
\item \underline{Transpose}: The transpose of a matrix $A$ of dimensions $m \times n$, denoted $A^{\top}$, is the matrix obtained by interchanging the rows and columns of $A$.
\item \underline{Symmetric}: A matrix $A$ is said to be symmetric if $A = A^{\top}$.
\end{itemize}

\begin{example}
Let $A$ and $B$ be the following matrices.
	\begin{align*}
	A = \begin{bmatrix}
	1 & 2 & 3 \\
	2 & 3 & 4 \\
	3 & 4 & 5 
	\end{bmatrix}
	\quad \text{and} \quad
	B = \begin{bmatrix}
	1 & 2 & 3 \\ 4 & 5 & 6
	\end{bmatrix} .
	\end{align*}
	\begin{enumerate}
	\item Find $B^\top$.
	\item Is $A$ symmetric?
	\end{enumerate}
\end{example}

\vfill


\noindent\underline{General facts about transpose}:
	\begin{itemize}
	\begin{multicols}{3}
	\item $(A^\top)^\top = A$.
	\item $(A + B)^\top = A^\top + B^\top$.
	\item $(cA)^\top = c A^\top$.
	\item $(AB)^\top = B^\top A^\top$.
	\item $(A^T)^{-1} = (A^{-1})^\top$.
	\item[\phantom{\textbullet}] \phantom{2}
	\end{multicols}
	\end{itemize}


\noindent\underline{General Facts about symmetric}:
Suppose $A$ and $B$ are matrices of the same size.
	\begin{itemize}
	\item If $A$ and $B$ are symmetric matrices, then so is $A + B$.
	\item If $A$ is symmetric, then $A$ is a square matrix and $cA$ is symmetric for any scalar $c$.
	\item $A^\top A$ and $A A^\top$ are symmetric matrices.
	\item If $A$ is an invertible symmetric matrix, then $A^{-1}$ is a symmetric matrix.
	\end{itemize}

\newpage

	\begin{example}
	Is the matrix
		\begin{align*}
		A = \begin{bmatrix}
		1 & 0 & 4 \\
		-4 & 0 & 6 \\
		2 & 0 & -10
		\end{bmatrix}
		\end{align*}
	invertible?
	\end{example}
	
	\vfill
	
	\noindent\underline{Remarks}: 
		\begin{itemize}
		\item Using row operations on the transposed matrix is equivalent to applying column operations to the original matrix.
		\item So, in general, what we learned to do with the rows of a matrix can also be done with the columns of a matrix.
		\item Taking column operations will be important when we will find the row space and column space of a matrix.
		\end{itemize}
\end{document}